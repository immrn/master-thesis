\subsection{Interview zur Einrichtung des neuen Verfahrens für zeitbasierte Einmalpasswörter}
\label{anh: interview einrichtung}

\begin{enumerate}
    \item Inwiefern haben Sie die Einrichtung als komplex oder leicht verständlich empfunden?
    \item Wie klar waren die einzelnen Schritte der Einrichtungen? An welchen Stellen wussten Sie nicht, was als nächstes zu tun war und wieso?
    \item Haben Sie (weitere) Anmerkungen oder Änderungsvorschläge für den gesamten Einrichtungsprozess?
\end{enumerate}
Vergleichen Sie die eben durchgeführte Einrichtung (Blue TOTP) mit der traditionellen Einrichtung.
\begin{enumerate}
    \setcounter{enumi}{3}
    \item Wie empfinden Sie den Mehraufwand bei Blue TOTP?
    \begin{enumerate}
        \item Was wären Gründe, Blue TOTP nicht zu nutzen? Wäre der Mehraufwand einer davon?
    \end{enumerate}
    \item Hätten Sie außerhalb der Studie die Einrichtung abgebrochen?
    \item Wie beurteilen Sie die Schwierigkeit der Einrichtung mit beiden Systemen? Fällt Ihnen die Einrichtung mit einem der Systeme leichter oder schwerer? Und inwiefern?
    \item Trauen Sie sich zu, einer anderen Person bei der \textbf{traditionellen} Einrichtung behilflich zu sein?
    \item Trauen Sie sich zu, einer anderen Person bei der Einrichtung von \textbf{Blue TOTP} behilflich zu sein?
    \item Im Vergleich zur traditionellen Einrichtung: Inwiefern wirkt Blue TOTP auf Sie authentisch, als wäre es ein fester Bestandteil der Einrichtung, oder aufgesetzt, als wäre es zusätzlich an die Einrichtung \glqq angebaut\grqq{}?
    \item Haben Sie noch weitere Anmerkungen zum Vergleich zwischen der traditionellen Einrichtung und der Einrichtung mit Blue TOTP?
\end{enumerate}

\subsection{Interview zur Authentisierung mit dem neuen Verfahren für zeitbasierte Einmalpasswörter}
\label{anh: interview authentisierung}

\begin{enumerate}
    \item Sie hatten für 6 Tage die Aufgabe, sich auf der Website anzumelden und eine simulierte Überweisung zu tätigen. Wie ist es Ihnen bei der Nutzung von Blue TOTP ergangen? Hat Ihnen etwas gefallen oder hat Sie etwas gestört?
    \begin{itemize}
        \item Falls negativ:
        Warum hat Sie XY gestört und würden Sie deswegen  Blue TOTP nicht verwenden?
        \item Was müsste geändert werden, damit Sie Blue TOTP gerne nutzen?
    \end{itemize}
    \item Haben Sie sich (weitere) Probleme notiert, die während der Studie aufgetreten sind?
    \item Wie waren Sie vor der Studie gegenüber 2FA eingestellt? Hat die Nutzung von Blue TOTP Ihre Einstellung gegenüber 2FA verändert (und wie)?
    \item Stellen Sie sich vor, Sie können (freiwillig) bei einem Dienst 2FA nutzen. Inwiefern wären Sie dazu bereit, die 2FA mithilfe von Blue TOTP zu nutzen?
    \item Stellen Sie sich vor, Sie müssen bei einem Dienst 2FA nutzen. Z.B. fordert Sie Ihr Arbeitgeber oder der Dienst selbst dazu auf. Inwiefern würden Sie diese Forderung akzeptieren, wenn Sie Blue TOTP dabei verwenden könnten?
    \item Das Ziel von Blue TOTP ist die 2FA mit zeitbasierten Einmalpasswörtern schneller und nutzerfreundlicher zu gestalten. Hatten Sie den Eindruck, dass das Blue TOTP beim Login langsamer oder schneller ist als das traditionelle Verfahren (und inwiefern)?
    \item \textit{Kurz erklären das Blue TOTP gewissermaßen einen Schutz gegen Phishing bietet (auf Phishing-Website sendet Blue TOTP keine Benachrichtigung, da es die Domain der Phishing-Website nicht kennt und so kein TOTP generieren kann).}\\
    Welche Eigenschaft ist für Sie am wichtigsten: die erweiterte Sicherheit, die Nutzerfreundlichkeit sowie schnellere Handhabung oder keins davon?
    \item Wie sehen Sie bei Blue TOTP das Verhältnis aus Aufwand und Nutzen zwischen der Einrichtung und der eigentlichen Nutzung beim Login?
    \begin{enumerate}
        \item Evtl. nachhaken: Überwiegt der Nutzen beim Login (schneller, nutzerfreundlicher) dem Mehraufwand bei der Einrichtung
    \end{enumerate}
    \item In welcher Reihenfolge sind Sie vorgegangen, wenn Sie sich anmelden und authentisieren mussten? (Bsp. erst  App geöffnet, dann Ext. mit App verbunden, dann Login auf Website, …)
    \item Die Blue TOTP App läuft im Hintergrund weiter, auch wenn Sie sie schließen (im Task Manager wegwischen). Haben Sie das wahrgenommen? Wodurch?
    \item Man kann diesen Hintergrundprozess von Blue TOTP auch beenden (Task Manager, Vordergrund Dienste, Blue TOTP, Stoppen). Wie oft haben Sie den Hintergrundprozess beendet und (ggf.) wieso?
    \item Hat die Verbindung zwischen der Android-App und der Browser-Extension einwandfrei funktioniert oder gab es zwischenzeitlich Probleme?
    \item Was ist der übliche Status der Bluetooth-Funktion Ihres \textbf{Computers}? (meist angeschaltet, nur angeschaltet, wenn gebraucht) Und hat die Nutzung von Blue TOTP dieses Verhalten geändert?
    \item Was ist der übliche Status der Bluetooth-Funktion Ihres \textbf{Smartphones}? (meist angeschaltet, nur angeschaltet, wenn gebraucht) Und hat die Nutzung von Blue TOTP dieses Verhalten geändert?
    \item Hat sich Ihrer Wahrnehmung nach der Akku Ihres Smartphones während der 6 tägigen Nutzungsphase schneller entladen als gewöhnlich?
    \item Welches Verfahren würden Sie bevorzugt nutzen, das traditionelle oder Blue TOTP? Warum?
    \item Wo sehen Sie noch Vor- oder Nachteile in Blue TOTP?
    \item Haben Sie weitere Anmerkungen oder Änderungsvorschläge für Blue TOTP?
\end{enumerate}