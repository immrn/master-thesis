Die Authentisierung dauerte im Median $8{,}5~s$ und im bereinigten Mittelwert 
$10,1~s$ (IRQ). Die kürzeste Messung zeigt, dass die Authentisierung innerhalb 
von ca. $4~s$ machbar ist. Es ist wichtig zu verstehen, dass die in dieser 
Arbeit gemessenen Authentisierungszeiten nur die Zeit von der erfolgreichen 
Eingabe des Passworts bis zur erfolgreichen Validierung der TOTPs meint. Die 
Blue TOTP Extension fordert das TOTP nur dann von der App an, wenn sie vor der 
Eingabe des Benutzernamens und Passworts mit der App verbunden ist. D.h., es 
gibt (bis auf die nicht identifizierbaren Messwerte mit Fallback-Option) keine 
Messwerte, bei denen der Proband sein Smartphone suchen musste oder während 
der Messung noch die Extension und die App verbinden musste. D.h. ,wenn man 
die Mediane oder oder bereinigten Mittelwerte betrachtet, dann bekommt man 
direkt eine Vorstellung davon, wie die Authentisierungszeiten sind, wenn die 
Blue TOTP App und Extension sich selbstständig wieder verbinden würden und der 
Nutzer Bluetooth an beiden Geräten (Computer und Smartphone) immer aktiviert 
lässt. In der Studie von \textcite{Reese} wurde die Authentisierungszeit 
genauso definiert wie hier. Dabei hatte das traditionelle TOTP-Verfahren einen 
Median von $15,1~s$, d.h. beinahe doppelt so viel Zeit wie Blue TOTP. Da Blue 
TOTP aus Nutzersicht stark dem Push-Verfahren ähnelt, ist auch dieser 
Vergleich interessant. Hier zeigt die Studie von \textcite{Resse} einen Median 
von $11{,}8~s$. Beim schnellsten Verfahren der Studie (Universal Second 
Factor) sind es $9{,}1~s$. Die Studie von \textcite{Reese} und die Studie in 
dieser Arbeit haben nur kleine Stichproben von 10 bis 12 Probanden pro 
Verfahren und es ist nicht möglich die Messungen von Reese mit den Messungen 
von Blue TOTP hinsichtlich statistischer Signifikanz zu untersuchen. Demnach 
kann man Hypothese (b) \glqq Der Prototyp des neuen Verfahrens für 
zeitbasierte Einmalpasswörter benötigt bzgl. der Authentisierungsvorgänge 
weniger Zeit als das gewöhnliche TOTP-Verfahren\grqq{} weder bestätigen noch 
falsifizieren. Aber es lässt sich vermuten, dass Blue TOTP hinsichtlich der 
Authentisierungszeit schneller als das traditionelle TOTP Verfahren ist.
\\\\
Man kann die Forschungsfrage, welche Auswirkungen die Implementierung (also 
Blue TOTP) auf die Task Completion Time hat, aus der Sicht betrachten, ob die 
Nutzer über den Verlauf der Studie kürzere Authentisierungszeiten erzielen 
konnten. Bezieht man sich auf den Median pro Tag ist eine leichte Tendenz zu 
erkennen, dass die Probanden bei täglicher Nutzung insgesamt schneller werden.
\\\\
Bei den Skalen des UEQ überragt Blue TOTP das traditionelle TOTP in jeder 
einzelnen Skala. Statistische Signifikanz zwischen den Systemen wird in den 
beiden Skalen Attraktivität und Originalität erreicht. Der Grund, dass Blue 
TOTP in der Attraktivität und den hedonischen Qualitäten (Stimulation und 
Originalität) einen besseren Mittelwert erreicht als das trad. TOTP, liegt 
vermutlich daran, dass Probanden das neue Verfahren als eine Innovation direkt 
wahrnehmen (z.B. durch weniger Nutzerinteraktionen oder Zeitersparnis) oder 
zumindest das Potential erkennen, dass Blue TOTP zukünftig erfüllen könnte. 
Die Werte von Blue TOTP für Attraktivität mit ca. $1{,}0$ und Stimulation mit 
$0{,}8$ sind nahe bzw. auf der Grenze zum neutralen Bereich. Allerdings muss 
man berücksichtigen, dass hier ein Verfahren zur 2FA untersucht wird und nicht 
ein marktfähiges Produkt. Wie stimulierend und attraktiv kann 2FA sein? 2FA 
ist etwas, das ein Nutzer tut, um mehr Sicherheit zu erzielen und nicht, weil 
er es aufgrund eines Verlangens nutzen will. Daher sind die anderen Skalen 
Durchschaubarkeit, Effizienz und Steuerbarkeit interessanter. Eine Erklärung 
dafür, wieso Blue TOTP eher durchschaubar ist, wäre, dass es weniger 
durchschauen lässt. Es klingt widersprüchlich, meint aber, dass der Nutzer die 
Einmalpasswörter fast nicht wahrnimmt und sich somit wenig Gedanken darüber 
macht. Bei den einzelnen Gegensatzpaaren zur Durchschaubarkeit haben die 
Probanden Blue TOTP vor allem als übersichtlicher, einfacher und leichter zu 
erlernen bewertet. Denn alles was der Nutzer durchschauen muss ist \glqq Wie 
verbinde ich App und Extension?\grqq{} und \glqq Wo bestätige ich die 
Notification/Anfrage?\grqq{}. Bluetooth sollte für die meisten Probanden eine vertraute Technologie sein. Und der kognitive Aufwand aus \glqq Wo finde ich 
mein TOTP in der TOTP-App?\grqq{} und \glqq Wie lange ist es noch gültig?\grqq{} 
entfällt bei Blue TOTP. Ein Unterschied von $0{,}5$ Punkten liegt zwischen der Effizienz 
der Systeme. Es wird vermutet, dass die Probanden Blue TOTP effizienter 
wahrnehmen, weil die neue Aufgabe, vor jeder Anmeldung App und Extension zu 
verbinden und eine Notification bestätigen, weniger aufwendig als die alte 
Aufgabe, die TOTP-App und das darin befindliche TOTP zu suchen, abzulesen und 
in der Website einzugeben. Theoretisch haben die fehlgeschlagenen Anmeldungen 
auch einen Einfluss auf die Effizienz. Bei $39\%$ der 59 Sitzungen gab es 
mindestens eine fehlgeschlagene Anmeldung. D.h. pro Nutzer gab es in den 6 
Nutzungstagen im Durchschnitt ca. $2{,}3$ fehlgeschlagene Anmeldungen. Rein 
quantitativ spricht es für eine schlechte Effizienz. Die Ursache liegt darin, 
dass Blue TOTP nur funktioniert, wenn erst App und Extension verbunden sind 
und dann die Anmeldung mit Benutzername und Passwort beginnt. Die Frage ist, 
wie sehr die Probanden die Fehlschläge noch in Erinnerung hatten. Denn nach 
deren Aussagen kann davon ausgegangen werden, dass die meisten Fehlschläge an 
den ersten Tagen der Studie passierten. Hinsichtlich der Steuerbarkeit haben 
die Probanden Blue TOTP hauptsächlich besser bewertet, weil sie es 
unterstützender empfunden haben als das trad. TOTP (siehe Abb. \ref{fig: 
studie ergebnisse auth ueq bt trad single} Gegensatzpaar behindernd/unterstützend). 
\\\\
Bei der SUS-Bewertung der beiden Systeme erhält Blue TOTP einen Median von 
$78$ Punkten und das traditionelle TOTP einen Median von $73$. Auch in den Q1 
und Q3 Quartilen sowie den Mittelwerten konnte eine Differenz zu Gunsten von 
Blue TOTP von ca. 5 Punkten festgestellt werden. Vergleicht man beide Mediane 
(Blue TOTP, trad. TOTP) mit dem Median zu TOTP aus der Arbeit von 
\textcite{Reese}, dann stellt man fest, dass Reese’ TOTP $88{,}8$ Punkte 
erreichte (also 15 Punkte mehr als trad. TOTP hier). Bei Reese befanden sich 
die Probanden nicht in der Situation, zwei verschiedene Systeme zu bewerten. 
Jeder Proband hat nur ein System bewertet. In der Studie von Blue TOTP wussten 
die Probanden allerdings, dass es darum ging TOTPs mit Blue TOTP zu verbessern 
und mussten ihre Bewertung zum trad. TOTP auf Basis ihrer Erfahrung bewerten, 
die je nach Proband unterschiedlich viele Tage zurückliegt. Die Frage, warum 
sich die Werte von Reese' SUS-Bewertung und der SUS-Bewertung der trad. TOTPs 
hier so sehr unterscheiden, kann nicht beantwortet werden. Aber es zeigt, dass 
die Ergebnisse von \textcite{Reese} bzgl. der SUS-Bewertung nicht als Referenz 
genutzt werden sollten.
\\\\
Hypothese (a) \glqq Der Prototyp des neuen Verfahrens für zeitbasierte 
Einmalpasswörter ist nutzerfreundlicher als das gewöhnliche 
TOTP-Verfahren\grqq{} kann aufgrund des SUS und UEQ nicht bestätigt werden, da 
keine statistische Signifikanz vorliegt, außer bei Attraktivität und 
Originalität des UEQ. Allerdings lässt sich auch hier anhand der einstimmigen 
Werte in allen Skalen des UEQ und des SUS sowie in den Aussagen der Probanden, 
dass mind. 5 von 10 Probanden Blue TOTP nutzerfreundlicher als das trad. TOTP 
empfinden, vermuten, dass Blue TOTP bzgl. der Authentisierung 
nutzerfreundlicher als das trad. TOTP ist.