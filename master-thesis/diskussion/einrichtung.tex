Die Einrichtung dauert im Mittel $3~min$ $48~s$. Der Median 
beträgt ca. $3~min$. Die kürzeste Messung zeigt, dass die 
Einrichtung innerhalb von $90~s$ durchführbar ist. Wie bei allen 
Werten der Studie sind Messwerte von 11 bzw. 10 Probanden meist 
nicht aussagekräftig. Aber man bekommt eine Vorstellung, in 
welcher Größenordnung sich die Ergebnisse befinden. Betrachtet man 
andere Studien, die die Einrichtung des traditionellen 
TOTP-Verfahrens gemessen haben, so schneidet Blue TOTP im Median 
besser ab als der Median von ca. $4~min~50~s$ (Google 
Authenticator) nach \textcite{Acemyan}. Die Studie von 
\textcite{Reese} zeigt dagegen deutlich kürzere Einrichtungszeiten 
mit einem Median von $84~s$. Es war zu erwarten, dass die 
Einrichtung mit Blue TOTP länger dauert als beim traditionellen 
Verfahren, da bei Blue TOTP wesentlich mehr Schritte nötig sind, 
bevor der QR-Code gescannt werden kann und der Nutzer das TOTP in 
die Website eingibt. Vergleicht man Blue TOTP mit den Einrichtungszeiten anderer 
2FA-Verfahren, ist der Unterschied noch größer. Es ist nicht konkurrenzfähig 
zu Zeiten wie $23{,}5~s$ für Push oder $44{,}0~s$ für den Universal Second Factor \autocite{Reese}
\\\\
Den höchsten Mittelwert unter den Skalen des UEQ erreichte die 
Effizienz mit $\bar{x} = 1{,}59$. Allerdings widerspricht der Wert 
gewissermaßen den Äußerungen der Probanden beim Interview. Hier 
wurde kritisiert, dass man zu häufig den Fokus zwischen der App 
und der Extension wechseln muss, was ein Indiz für mangelnde 
Effizienz ist. Auch hinsichtlich der Task Completion Time für die 
Einrichtung ist Blue TOTP nicht sonderlich effizient. Dagegen 
stimmt der niedrige Mittelwert von $\bar{x} = 1{,}11$ für die 
Durchschaubarkeit mit den Aussagen der Probanden und den 
Beobachtungen überein. Bspw. war es für die meisten unerwartet, 
eine Anleitung auf der App zu sehen, wenn man auf den 
Kamera-Button tippt. Oft wurden die Anleitungen bestätigt, ohne 
sie gelesen zu haben. Für die Probanden war nicht durchschaubar, 
wieso kein Echtzeitbild der Smartphone-Kamera zu sehen war, bzw. 
war es für einige zu viel Text. Genauso wenig hat Blue TOTP 
kommuniziert, wieso es bspw. einer Extension bedarf oder wofür 
Bluetooth genutzt wird. Ist ein System nicht durchschaubar, fühlen 
sich die Probanden vermutlich auch nicht sonderlich 
zuversichtlich. Zumindest lässt das der Durchschnittswert der 9. 
Aussage der SUS vermuten. Hier vergaben die Probanden im Mittel 
eine nahezu neutrale Antwort, ob sie sich sehr zuversichtlich 
gefühlt haben. Diese geringe Zuversichtlichkeit äußert sich auch 
darin, dass der Versuchsleiter bei den meisten Einrichtungen 
zumindest einen Hinweis geben musste, was als nächstes zu tun ist. 
Es sei noch die Steuerbarkeit erwähnt. Der Mittelwert des UEQ 
ergibt hier $1{,}25$. Bis auf den Bug, der bei zwei Einrichtungen 
erschien, gab es keine ungewollten Zwischenfälle, die 
Steuerbarkeit des System beeinträchtigen würden. Der Autor 
vermutet, dass der bereits beschriebene häufige Fokuswechsel 
zwischen Extension und App das Gefühl der Kontrolle entreißt, da 
man mit der Website einbezogen auf drei User Interfaces achten 
muss.
\\\\
Der SUS ergab eine Punktzahl von $77{,}5$ im Median. Somit gilt 
nach dem Grenzwert, ab wann ein System als akzeptabel 
nutzerfreundlich angesehen wird, dass die Einrichtung von Blue 
TOTP akzeptabel ist \autocite{BrookeRetro}. In der Arbeit von 
\textcite{Acemyan} erreichte der Google Authenticator (TOTP) für 
die Einrichtung ca. 52 SUS-Punkte. Allerdings scheint dies keine 
annähernd repräsentative SUS-Bewertung für das traditionelle 
TOTP-Verfahren zu sein, wenn schon die Zeitmessung deutlich von 
den Ergebnissen der Studie von \textcite{Reese} abweicht. 
Interessant ist auch der Fakt, dass nur drei Probanden bei der 
Aussage 4, dass sie glauben, Hilfe zu benötigen, eher bis 
vollständig zustimmen und alle anderen eher bis vollständig 
dagegen stimmen. Zusammen mit dem Zitat von P1 \glqq Wenn es 
einmal weiß, geht es\grqq{} und den beschriebenen Beobachtungen 
ist ersichtlich, dass das System noch ausbaufähig ist, um selbsterklärend zu werden. Die SUS-Bewertungen der Probanden scheinen 
auch ihren Empfindungen bzgl. der Nutzerfreundlichkeit von Blue 
TOTP übereinzustimmen, da die Standardabweichung zwischen echter 
und empfundener Bewertung nur $7{,}7$ beträgt. Schließlich muss 
man bedenken, dass die Zuordnung der sieben möglichen Antworten 
auf je eine feste Zahl trifft \autocite[121]{SUS11} und somit eine 
Diskretisierung stattfindet, durch die praktisch immer eine 
gewisse Standardabweichung resultiert.
\\\\
Hypothese (a) \glqq Der Prototyp des neuen Verfahrens für 
zeitbasierte Einmalpasswörter ist nutzerfreundlicher als das 
gewöhnliche TOTP-Verfahren\grqq{} kann bzgl. der Einrichtung nicht 
quantitativ über SUS und UEQ bestätigt werden, weil die Probanden 
keinen SUS und UEQ zur Einrichtung des traditionellen Verfahrens 
ausfüllen mussten. Zu wenige Studien liefern dahingehend 
Vergleichswerte, zumal deren Experimente sich in Details wie der 
verwendeten App oder dem Webservice unterscheiden. Das gleiche 
gilt für die qualitativen Aussagen, die nur einseitig bzgl. Blue 
TOTP abgefragt wurden. Bzgl. der Forschungsfrage, wie die 
Implementierung auf die Nutzerfreundlichkeit und Nutzererfahrung 
ausprägt, lässt sich anhand der UEQ- und SUS-Bewertungen erkennen, 
dass die Einrichtung solide aber noch nicht ausreichend 
nutzerfreundlich ist und eine bessere Nutzererfahrung bieten 
könnte. Die qualitativen Aussagen und Beobachtungen identifizieren 
einige Probleme der Einrichtung und zeigen, dass sie in ihren 
Grundzügen funktioniert, aber noch viel Potential zur Verbesserung 
bietet.