Für die weitere Forschung wird empfohlen, zunächst die Fehler von Blue TOTP zu 
beheben und einigen Änderungsvorschlägen der Probanden nachzukommen.

\paragraph*{Entwicklung von Blue TOTP (Einrichtung)}
\mbox{} \vspace{0.1cm} \\
Besonders die Einrichtung kann optimiert werden. Es sollte der gesamte Ablauf 
der Einrichtung überdacht werden. Dabei sollte es keine Rolle spielen, von 
welchem Punkt der Nutzer startet. Blue TOTP ist momentan so konzipiert, dass 
es dem Nutzer alles erklärt, was zu tun ist, bevor er den QR-Code scannen 
darf. Aber die Studie hat eindeutig gezeigt, dass es nicht nötig ist. Der 
Ausgangspunkt, dass ein Nutzer bereits die 2FA-Einrichtung auf einer Website 
gestartet hat, verlangt, dass Blue TOTP keine Anleitungen mehr dazu liefert, 
sondern ihn direkt den QR-Code scannen lässt und ihn dann bspw. in der App 
(und nicht in der Extension) fragt, wie sein Benutzername lautet. Der 
Benutzername kann dort ebenfalls schon vorausgefüllt sein, falls die Extension 
einen erkannt hat. Es ist auch eine Überlegung wert, die Anleitung nicht nur 
zu kürzen, sondern das, was davon noch übrig ist, gänzlich auf dem Smartphone 
anzuzeigen, da der Nutzer ohnehin die App fokussiert. Das lässt sich auch gut 
mit dem nächsten Punkt, dem automatischen Wiederverbinden, kombinieren. Die 
Extension hätte dann aus Nutzersicht nur die Hauptfunktion, vor der ersten 
Verbindung zwischen App und Extension nach dem Smartphone zu scannen, so dass 
der Nutzer es auswählt. Idealerweise führt der Nutzer noch einmalig Schritte 
durch, die die Verbindung auch tatsächlich authentisieren und sicher machen (z.
B. einen QR-Code in der Extension scannen o.ä.). Ist die automatische 
Wiederverbindung implementiert, dann kann die Extension wirklich umfungiert 
werden, so dass der Nutzer sie für die Einrichtung eines 2FA-Verfahrens 
überhaupt nicht benötigt. Es ist evtl. vorteilhaft, wenn Blue TOTP 
kommuniziert, dass es mit Bluetooth funktioniert und diese Verbindung sicher/
Ende-zu-Ende-verschlüsselt ist und es sicherer als andere 2FA-Verfahren ist 
(zumindest die nicht Phishing-resistenten).

\paragraph*{Entwicklung von Blue TOTP (Authentisierung)}
\mbox{} \vspace{0.1cm} \\
Auch wenn dann eine automatische Wiederverbindung implementiert ist, sollte es 
für den Nutzer möglich sein, erst Benutzername und Passwort einzugeben und 
dann zu realisieren, dass er Bluetooth auf seinem Smartphone oder Computer 
aktivieren muss, damit sich Extension und App verbinden. Einige Teilnehmer 
hatten vorgeschlagen, dass das TOTP, nachdem es automatisch in die Website 
übertragen wurde, automatisch bestätigt wird. Im Hinblick auf einen neuen 
Standard für TOTPs wie es das Konzept in Kap. \ref{sec: funktionsweise 
konzept} vorschlägt, könnte man überlegen, das TOTP-Eingabeelement gänzlich zu 
entfernen und für einen evtl. Fallback zu verwenden. Wie bereits am Ende des 
Kap. \ref{sec: konzept fallback} angesprochen, sollte auch die 
Sicherheitslücke im Konzept geschlossen werden, die die Unwissenheit des 
Nutzers ausnutzt, und ihn mit einem eigenen QR-Code für den Fallback zur 
Preisgabe seines TOTPs bringt. Auch der Fallback von Blue TOTP sollte dann 
eine Lösung finden, Phishing-resistent und nutzerfreundlich zu sein. Des Weiteren sollte eine Lösung gefunden werden, wie man auf bessere Art und Weise einen Hintergrundprozess dauerhaft auf dem Smartphone betreibt (zumindest für Android, zu iOS kann keine Aussage gemacht werden). 

\paragraph*{Forschung}
\mbox{} \vspace{0.1cm} \\
Für die Forschung wird empfohlen, nach der Weiterentwicklung von Blue TOTP, 
die Studie ähnlich durchzuführen. Dabei sollte eine zweite Gruppe an Probanden 
rekrutiert werden, die das traditionelle TOTP-Verfahren testet und vor allem 
Messwerte zur Einrichtungs- sowie Authentisierungszeit liefert. Bei der neuen 
Studie sollte vorher festgelegt werden, ob die Fallback-Option von Blue TOTP 
entfernt wird oder ob man stattdessen in allen Software-Komponenten die 
Nutzerinteraktionen protokolliert. So könnte man später ggf. Anmeldungen, die 
mit dem Fallback getätigt wurden, aus Ergebnissen entfernen.
Es könnte auch ratsam sein die Studie länger durchzuführen, z.B. für zwei 
Wochen, um zu sehen, ab welchen Wert es für die Authentisierungszeit eine 
Sättigung gibt, wenn es darum geht, ob Nutzer durch die tägliche Verwendung 
schneller in der Handhabung von Blue TOTP werden.