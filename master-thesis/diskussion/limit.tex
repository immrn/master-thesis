Die Studie fokussiert sich weniger darauf, die Hypothesen mithilfe 
statistischer Signifikanz zu bestätigen oder zu falsifizieren. Mit einer 
kleinen Stichprobenmenge von 11 (Einrichtung) bzw. 10 (Authentisierung) 
Probanden ist es in diesem Kontext zu erwarten, keine statistische 
Signifikanz zu erreichen. Der Fokus liegt eher darauf, ob potentiell eine 
Tendenz bzgl. der Hypothesen zu erkennen ist und welche Stärken und 
Schwachstellen Blue TOTP aufweist und mit welchen Schritten man es verbessern 
kann.
\\\\
Bei der Einrichtung kam es bei zwei Teilnehmern vor, dass das initiale TOTP 
nicht von der App angezeigt wurde. Auch traten Bugs bei der Authentisierung 
auf, bei denen die Teilnehmer das System trotzdem noch nutzen konnten und nur 
ihr Nutzererlebnis etwas negativ beeinflusst wurde. Selbstverständlich sollte 
mehr Zeit in Testing investiert werden, um Bugs auszuschließen. Dennoch 
sollten die wenigen Bugs, die aufgetreten sind, das Ergebnis nicht so sehr 
beeinflusst haben, da die Probanden dahingehend verständnisvoll wirkten.
\\\\
Dadurch, dass die Fallback-Option den Probanden zur Verfügung stand, haben sie 
diese teilweise genutzt und somit hauptsächlich die Zeitmessungen verfälscht. 
Der Vorteil ist, dass man mit viel Sicherheit annehmen kann, dass die 
Fallback-Verwendungen die Authentisierungszeiten bzgl. des Medians und 
Mittelwerts positiv verzerrt haben, also den Median bzw. Mittelwert größer 
erscheinen lassen, als eigentlich ist. D.h. der Median und Mittelwert 
fungieren eher als eine obere Grenze für den tatsächlichen Wert. Umgekehrt 
wäre schlechter für die Studie: also die Mittelwerte wären in Wahrheit größer, 
weil der Fallback i.d.R. kürzer dauert als der Weg mit der 
Bluetooth-Funktionalität.
\\\\
Der Fakt, dass Probanden nicht innerhalb der Studie das traditionelle TOTP-Verfahren verwendet haben, wurde bereits im vorangegangenen Unterkapitel diskutiert.