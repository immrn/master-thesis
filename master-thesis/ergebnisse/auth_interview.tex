Am Ende der Abschlussveranstaltung hat der Versuchsleiter mit den Probanden ein 
Interview zur eigentlichen Nutzung von Blue TOTP geführt. Wie bei der Einrichtung 
lassen sich auch für die Authentisierung die Aussagen der Probanden in 
Themengebiete unterteilen. Einige Aussagen können kategorisiert und somit 
statistisch festgehalten werden (siehe Tab. \ref{tab: exit meeting statistic}).
\\\\
\paragraph*{Nutzererfahrung}
\mbox{} \vspace{0.1cm} \\
Zu Beginn des Interviews wurden die Probanden befragt, wie es ihnen bei der Nutzung 
von Blue TOTP ergangen ist und welche Probleme aufgetreten sind. Aus den Antworten 
lässt sich ableiten, dass 6 der 10 Probanden bzgl. der Nutzung eher positiv 
beeindruckt waren. Ein Proband hat sich gefreut, dass er im Vergleich zum 
traditionellen Verfahren nicht mehr auf das nächste TOTP warten muss, wenn das 
aktuelle nur noch wenige Sekunden gültig ist. Ein anderer Proband berichtete, dass 
er sich die App vollständig angeschaut und die Fallback-Option entdeckt hat. Diese 
verwendete er an zwei Tagen. Er hat sich dann aber doch wieder für die 
Bluetooth-Option entschieden, da er feststellte, dass ihm die Anmeldung so viel 
leichter fällt. Dagegen äußerte ein Proband, dass sich Blue TOTP für ihn nicht 
lohnt, da Bluetooth an seinem Smartphone für gewöhnlich nie aktiviert ist. Er 
kritisiert, dass er immer erst Bluetooth aktivieren und dann die beiden Geräte 
verbinden muss. So würde sich Blue TOTP für ihn nicht lohnen, mitunter 
weil er bei den meisten Diensten durch Session Cookies sowieso angemeldet bleibt. 
Auch technische Fehler traten auf, wie z.B. der Bug, dass in der Extension beim 
Bluetooth-Scan das Smartphone mehrmals angezeigt wurde. Dies geschah, wenn der PC zwischen zwei Verbindungsvorgängen 
nicht heruntergefahren wurde. Auch kam es vor, dass die App anzeigt, sie sei 
verbunden, obwohl sie es nicht ist. Ein Proband berichtete, dass er an einem Tag 
die Extension und App nicht verbinden konnte und dann den Fallback entdeckt und 
verwendet hat. Neben diesen Schwierigkeiten haben die Probanden einige störende 
Faktoren zusammengetragen. 5 Probanden hatten am ersten Tag der Nutzungsphase die 
Reihenfolge vertauscht, wie sie vorgehen müssen. Sie haben zuerst Benutzername und 
Passwort auf der Website eingegeben und dann realisiert, dass sie vorher App und 
Extension verbinden müssen und nicht nachher. D.h. sie mussten die Website erneut 
aufrufen und wieder Benutzername und Passwort eingeben. Erst dann sahen sie die 
Notification auf dem Smartphone. Zwei Probanden berichteten, dass ihnen der Scan zu 
lang dauert. Hier sei erwähnt, dass es teilweise bis zu $5~s$ oder länger dauern kann, bis die Extension das Smartphone gefunden hat. Hinsichtlich der App 
kritisieren zwei Probanden, dass man die dauerhafte Notification (Android 
Foreground Service) nicht wegwischen konnte (abhängig von Hersteller und Android Version).
\\\\
\paragraph*{Nutzungsverhalten}
\mbox{} \vspace{0.1cm} \\
Des Weiteren wurden die Probanden gefragt, unter welchen Bedingungen sie Blue TOTP 
gern nutzen würden. Zuerst wurde die Frage gestellt, inwiefern sie Blue TOTP bei 
einem Dienst nutzen würden, der ihnen freistellt, ob sie einen Zwei-Faktor-Schutz 
verwenden wollen. 8 Probanden würden dafür Blue TOTP verwenden, aber einige 
unterscheiden zwischen sicherheitsrelevanten und nicht sicherheitsrelevanten 
Diensten. 4 Probanden würden es nur bei sicherheitsrelevanten Diensten nutzen. Ein 
Proband gab sogar an, es nur bei nicht sicherheitsrelevanten Diensten nutzen zu 
wollen, weil er bspw. beim Banking das Gefühl hat, dass ein Gerät ohne 
Internetzugang sicherer ist. Danach wurde die Frage analog zu dem Fall gestellt, 
wenn der Dienst die Probanden zwingen würde, einen Zwei-Faktor-Schutz zu nutzen. 7 
Teilnehmer gaben an, diese Aufforderung eher zu akzeptieren, wenn sie Blue TOTP 
nutzen dürften. Einige begründeten ihre Aussagen damit, dass Blue TOTP im Vergleich 
zum trad. TOTP für sie einfacher und schneller in der Nutzung sein. Eine weitere 
Frage war, wie die Probanden bei der Anmeldung vorgegangen sind. Die Frage zielt darauf ab, 
ob sie erst die Website aufrufen oder erst die Extension und App miteinander verbinden. 
Teilweise sind sie von Tag zu Tag unterschiedlich vorgegangen, d.h. beide Fälle kamen vor. Von 
Interesse war auch, wie die Probanden Bluetooth vor und während der Studie genutzt 
haben: War es immer aktiviert oder wurde es nur dann aktiviert, wenn man es braucht? 
Die Antworten sind in Tab. \ref{tab: exit meeting statistic} zu sehen unter \glqq 
Bluetooth im PC/Smartphone vor/während/ der Studie\grqq{}.
\begin{table}
    \begin{center}
    \resizebox{0.7\textwidth}{!}{%
    \begin{tabular}{| l | r | c |}
        \hline
        \textbf{Kriterium} & \textbf{Antwort} & \textbf{Anzahl} \\
        \hline \hline
        \multirow{3}{*}{\parbox{0.4\linewidth}{Empfinden über Zeitaufwand\\(Blue TOTP vs. trad. TOTP)}} & schneller & 6 \\
        & gleich schnell & 2 \\
        & langsamer & 2 \\
        \hline
        \multirow{3}{*}{\parbox{0.4\linewidth}{Empfinden über  Nutzerfreundlichkeit\\(Blue TOTP vs. trad. TOTP)}} & besser & 5 \\
        & gleich & 3 \\
        & keine Angabe & 2 \\
        \hline
        \multirow{2}{*}{\parbox{0.4\linewidth}{Überwiegt Aufwand oder Nutzen}} & Nutzen & 9 \\
        & Aufwand & 1 \\
        \hline
        \multirow{2}{*}{\parbox{0.4\linewidth}{Mind. 1x die Hintergrund-Funktionalität der App genutzt}} & Ja & 4 \\
        & Nein & 6 \\
        \hline
        \multirow{2}{*}{\parbox{0.4\linewidth}{Bluetooth im PC vor der Studie}} & immer aktiviert & 7 \\
        & nur aktiviert, wenn gebraucht & 3 \\
        \hline
        \multirow{2}{*}{\parbox{0.4\linewidth}{Bluetooth im PC während der Studie}} & immer aktiviert & 9 \\
        & nur aktiviert, wenn gebraucht & 1 \\
        \hline
        \multirow{2}{*}{\parbox{0.4\linewidth}{Bluetooth im Smartphone\\vor der Studie}} & immer aktiviert & 1 \\
        & nur aktiviert, wenn gebraucht & 9 \\
        \hline
        \multirow{2}{*}{\parbox{0.4\linewidth}{Bluetooth im Smartphone\\während der Studie}} & immer aktiviert & 2 \\
        & nur aktiviert, wenn gebraucht & 8 \\
        \hline
        \multirow{3}{*}{\parbox{0.4\linewidth}{Bevorzugung}} & Blue TOTP & 7 \\
        & beides gleichermaßen & 2 \\
        & trad. TOTP & 1 \\
        \hline
    \end{tabular}}
    \end{center}
    \caption[Statistische Ergebnisse der abschließenden Interviews]{Statistische Ergebnisse der abschließenden Interviews über den Anmeldungsprozess mit Blue TOTP bzw. mit dem trad. TOTP; Die Anzahl sagt aus wie viele Probanden der Studie zu einer Antwort tendieren}
    \label{tab: exit meeting statistic}
\end{table}
Es ist zu sehen, dass zwei von potenziell drei Probanden ihr Verhalten geändert 
haben, bzgl. der Sache, ob sie Bluetooth am Computer immer aktiviert lassen, 
anstatt es nach dem Gebrauch wieder zu deaktivieren. Analog zum Smartphone hat nur 
einer von potentiell 9 Probanden sein Verhalten geändert und Bluetooth immer 
aktiviert gelassen. Er meinte, dass die Nutzung von Blue TOTP für ihn so am 
angenehmsten sei.

\paragraph*{Präferenz}
\mbox{} \vspace{0.1cm} \\
Außerdem wurden die Probanden hinsichtlicher ihrer Präferenzen befragt. Zum einen 
gab es die Frage, welches System (Blue TOTP oder traditionelles TOTP) sie 
bevorzugen und ob bei ihnen bzgl. Blue TOTP der Aufwand (umfasst auch die 
Einrichtung) oder der Nutzen überwiegt. Die Ergebnisse sind in 
Tab. \ref{tab: exit meeting statistic} zu sehen. Ein Proband meinte, für ihn überwiege der Aufwand dem 
Nutzen. Das lag daran, dass es ihn störte, nur für eine Anmeldung Bluetooth am 
Smartphone zu aktivieren und das Smartphone mit der Extension zu verbinden. $70\%$ 
der Probanden bevorzugen Blue TOTP gegenüber dem traditionellen TOTP, weil es ihrer 
Wahrnehmung nach schneller oder nutzerfreundlicher ist (siehe auch 
Tab. \ref{tab: exit meeting statistic} \glqq Empfinden über Zeitaufwand/
Nutzerfreundlichkeit\grqq{}). Zusätzlich wurde den Probanden erklärt, dass Blue 
TOTP versucht, schneller und einfacher in der Handhabung zu sein, aber dass es auch 
entgegen anderer 2FA-Verfahren Phishing-resistent ist. Bei der Frage, welche 
Eigenschaft ihnen bei Blue TOTP wichtiger ist, stimmten 6 Probanden für die 
Sicherheit und 3 für die schnellere Handhabung sowie gesteigerte 
Nutzerfreundlichkeit.

\paragraph*{Technisches}
\mbox{} \vspace{0.1cm} \\
Bezüglich des Energieverbrauchs auf dem Smartphone konnte kein Proband einen 
Unterschied erkennen, inwiefern der Akku sich schneller mit der (im Hintergrund) aktiven Blue TOTP App 
entladen hat. Da die App im Hintergrund lief, wurden die Teilnehmer auch gefragt, 
ob sie bemerkt haben, dass die App immer aktiv ist (solange Android sie nicht 
beendet) und ob sie die App teilweise auch nicht geöffnet haben, um die Extension 
und die App zu verbinden. 4 Probanden machten mind. einmal Gebrauch von der  
Hintergrund-Funktionalität.

\paragraph*{Änderungsvorschläge}
\mbox{} \vspace{0.1cm} \\
Drei Probanden wünschen sich, dass das TOTP automatisch bestätigt wird, nachdem es 
von der Extension automatisch in das Eingabeelement eingefügt wurde. Auf Nachfrage 
haben es nahezu alle Teilnehmer befürwortet, wenn sich Extension und App 
automatisch verbinden würden. Die Stimmen dagegen meinten, dass sie sich unwohl 
fühlen würden (vermutlich ein Gefühl von Kontrollverlust). Ein Proband merkte an, 
dass es egal sein sollte, ob man erst Benutzername und Passwort eingibt und dann 
die Extension mit der App verbindet oder andersrum. Das bestätigt auch die 
Erkenntnis beim Nutzungsverhalten, da viele Teilnehmer den Fehler (erst 
Benutzername \& Passwort, dann Bluetooth-Verbindung) am ersten Tag machten. Ein 
Teilnehmer wünscht sich, dass sich die Extension von selbst öffnet, sobald man sie 
braucht (in Chrome ist dies technisch nicht möglich). Es ist ein helles Design für 
die Extension gewünscht.