Mit der Authentisierung ist das Erlangen des TOTPs und die Übergabe an die Website 
gemeint. Genau genommen ist die Eingabe von Benutzername und Passwort ebenso Teil der 
Authentisierung, aber das ist für die folgend präsentierten Daten nicht von Relevanz. 
Es sind Daten von 10 Probanden verfügbar, da ein Proband Blue TOTP innerhalb der 
Nutzungsphase nicht mit Bluetooth genutzt hat, sondern mit dem Fallback. Einzelne Authentisierungsvorgänge 
wurden nach Angaben der Probanden 
mit der Fallback-Option (TOTP ablesen und eintippen) getätigt, sind aber nicht mehr 
zuverlässig identifizierbar, um sie zu entfernen.