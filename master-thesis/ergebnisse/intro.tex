Nach dem Abschluss der Studie wurden die Ergebnisse für den Prototyp  \glqq Blue TOTP\grqq{} sowie einige Angaben zu dem traditionellen Verfahren für zeitbasierte Einmalpasswörter (TOTP) aufbereitet. Für die Daten zur Einrichtung gab es keine Werte wie Ausreißer oder inkonsistente Angaben in Fragebögen, die entfernt werden mussten. Seitens der Daten zur Authentisierung, also dem eigentlichen Anwendungsfall des Prototypen, wurde der Proband P9 entfernt. Grund dafür war, dass der Prototyp zur Authentisierung nicht mit der Bluetooth-Funktionalität genutzt wurde, sondern ausschließlich mit der Fallback-Option.
\\\\
Die Ergebnisdarstellung beginnt mit der Einrichtung, also der Einrichtungszeit, den Ergebnissen des User Experience Questionnaire (UEQ) und System Usability Scale (SUS) Fragebogens sowie Aussagen aus den Interviews. Anschließend werden in der gleichen Struktur die Ergebnisse zur Authentisierung dargestellt. Bei der Authentisierung wird neben dem Prototypen auch das traditionelle TOTP-Verfahren betrachtet.
\\\\
Alle Ergebnisse und deren Herstellungsweise sind im Repository (siehe Kap.~\ref{sec: implementierung}) aufgeführt.
