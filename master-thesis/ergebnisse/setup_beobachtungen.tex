Während die Probanden die Einrichtung durchgeführt haben, hat der Versuchsleiter sie dabei beobachtet und Auffälligkeiten protokolliert. Zu Beginn hat die Website immer den QR-Code und eine 
allgemeingültige Anleitung angezeigt, die auch für das normale TOTP-Verfahren 
funktioniert.
\\\\
Jeder Proband hat damit begonnen, den angezeigten QR-Code zu scannen. Ungefähr 
die Hälfte hat versucht, den QR-Code mit der Scan-Funktion von Android zu 
scannen, anstatt dies direkt über die App zu tun. Allerdings unterstützt die 
Blue TOTP App es nicht, dass man sie über die Scan-Funktion von Android öffnet.
Einige wussten nicht, wie sie beginnen sollen. Später stellte sich heraus, dass 
sie den Kamera-Button nicht ausreichend wahrgenommen haben.
Nachdem man den Kamera-Button antippt, erscheint eine Anleitung, wie man die 
App mit der Extension verbindet. Mehrere Probanden haben diese Anleitung nach 
wenigen Sekunden verlassen, indem sie den OK-Button gedrückt haben. Später 
stellte sich heraus, dass sie anstatt eines Textes das Echtzeit-Kamerabild 
erwartet hatten, um den QR-Code zu scannen.
Der Anleitung in der App nach hat man dann die Extension und die App verbunden. 
Einem Teilnehmer war unklar, was als nächstes zu tun ist. In der App stand als 
letzter Hinweis, dass man in der Extension die Einrichtung starten soll (also 
einen Button drücken).
\\\\
Sobald App und Extension verbunden sind und man in der Extension den Button zur 
Einrichtung drückt, sieht man eine geführte Anleitung.
Die Schritte 1 und 2 auf dem zweiten 
der insgesamt vier Anleitungs-Screens (siehe Anh. \ref{anh: blue totp ext screens} Abb. \ref{fig: blue totp ext screenshot anleitung 2}) waren bei jedem Probanden bereits 
erfüllt. Die Website zeigte zu diesem Punkt schon den QR-Code an. Allerdings 
waren einige davon verwirrt, wieso die Extension ihnen diese Anweisung gibt. 
Ein Proband dachte sogar, er müsse sich nochmal beim Webdienst der Studie 
anmelden. Das Problem war, wie einige dann berichteten, dass sie bei Schritt 1 
bzw. 2 verwirrt waren und deshalb nicht weiter die Anleitung (also Schritt 3) 
verfolgten.
\\\\
Am Ende dieser Anleitung wurde der Proband dazu aufgefordert, den Kamera-Button 
in der App zu betätigen, um den QR-Code zu scannen. Wenige Probanden klickten 
daraufhin auf den Zurück-Button in der Extension. Hat man dann den QR-Code mit 
der App gescannt, erschien in der App ein Screen, der immer das aktuelle TOTP 
anzeigt. Zweimal kam es vor, dass die App fälschlicherweise nicht diesen Screen 
angezeigt hat.
\\\\
Allgemein wurde noch beobachtet, dass die Extension oft den QR-Code überdeckt. 
Jeder Proband wusste aber, dass er die Extension schließen muss, um dies zu 
umgehen.