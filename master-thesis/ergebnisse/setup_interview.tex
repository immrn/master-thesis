Nach der Einrichtung und den Fragebögen wurde mit den Teilnehmern ein Interview 
geführt. Die Fragen bzw. die Aussagen der einzelnen Probanden wurden in 
Themengebieten zusammengefasst, um besser Stärken oder Schwächen zu 
identifizieren. Angaben dazu, dass ein Teil der Probanden die gleiche Meinung 
vertreten oder eine ähnliche Erfahrung hatten, impliziert nicht, dass die 
restlichen Probanden die gegenteilige Meinung oder Erfahrung teilen. Es werden 
auch einige Einzelaussagen festgehalten, die für Erkenntnisse oder 
Optimierungen bzgl. der Studie sowie Blue TOTP von Interesse sein könnten.

\paragraph*{Komplexität und Navigation}
\mbox{} \vspace{0.1cm} \\
Den Probanden wurde die Frage gestellt, wie komplex oder verständlich sie die 
Einrichtung empfanden und wie klar die einzelnen Schritte in den Anleitungen 
waren. Dabei antworteten 6 von 11 Probanden, dass der gesamte Prozess 
verständlich sei und 4 von 11, dass ihnen meistens bewusst war, was als nächstes 
zu tun ist.
\\\\
\hspace*{6mm} P1: \textit{\glqq Wenn man es einmal weiß, geht es.\grqq{}}
\\\\
Jemand beschrieb die Einrichtung als ungewohnt, da die traditionelle Einrichtung 
weniger Schritte verlangt. Auch stellten sich Teilnehmer die Frage, wieso man 
App und Extension per Bluetooth verbindet. Ein Proband meinte, die Anleitung sei 
gut formuliert, ein anderer beschrieb sie als unterstützend. Dagegen wurde oft 
geäußert, dass die Anleitung in der Extension verwirrend war. Besonders der 
zweite Screen sorgte wie in Kap. \ref{sec: ergebnisse studie beobachtungen} 
beschrieben, für Verwirrung. Beim ersten Screen hinterfragte ein Teilnehmer 
(zurecht), wozu dieser wichtig sei. Des Weiteren war einmal unklar, wie man die 
Extension öffnet, und ein Proband merkte an, dass Begriffe wie \glqq Extension\grqq{} für 
ihn zu unbekannt seien. Er fügt hinzu, zu Beginn der Anleitung in der App eine 
Animation zu zeigen, wie man auf dem PC die Extension im Browser findet. Stark 
kritisiert wurde der Wechsel von der App zur Extension und wieder zurück. Ein 
Teilnehmer wusste z.B. nicht, dass er wieder die Anleitung auf der App verfolgen 
muss, sobald er die App und Extension per Bluetooth verbunden hat. Ein 
Teilnehmer wünschte sich, dass die Anleitung in die Website integriert ist.

\paragraph*{Texte und Symbolik}
\mbox{} \vspace{0.1cm} \\
Einige Probanden gaben an, dass sie die Anleitungen nicht richtig durchgelesen 
haben, da es zu viel Text war oder sie keinen Text erwartet hatten. So war es 
ungewöhnlich, auf den Kamera-Button in der App zu tippen und daraufhin Texte zu 
sehen. Teilweise wurde der Text mit \glqq Ok\grqq{} bestätigt, ohne ihn gelesen zu haben. 
Die Probanden schlagen vor, den Kamera-Button in einen Button mit Plus-Symbol 
oder mit dem Text \glqq Einrichtung starten\grqq{} zu ändern. So würde man nicht das 
Echtzeitbild der Kamera erwarten und wäre eher dafür bereit, einer Anleitung zu 
folgen. Ein Proband meinte, die Anleitung in der Extension nicht wirklich 
gelesen zu haben, da er einfach den blau hervorgehobenen Buttons folgte. 
Außerdem wurde der Vorschlag gemacht, die Anleitung weitestgehend zu 
verbildlichen (bewegte Bilder nicht ausgeschlossen).

\paragraph*{Vergleich zum traditionellen Verfahren}
\mbox{} \vspace{0.1cm} \\
Bei der Frage, wie die Probanden den Mehraufwand von Blue TOTP im Vergleich zum 
traditionellen TOTP-Verfahren sehen, antworteten 5 von 11, dass der Mehraufwand 
gering ist. Zwei weitere gaben an, dass es deutlich aufwendiger ist. Ein 
Teilnehmer meinte, dass er bereit sei, den Mehraufwand auf sich zu nehmen, wenn 
er wüsste, dass es später einen Mehrwert hätte. Der Mehraufwand und die 
Tatsache, eine Extension zu verwenden, wurde einzeln als ein Grund genannt, Blue 
TOTP nicht zu nutzen. Ansonsten sahen die Teilnehmer keinen ernsthaften Grund. 
Bei der Frage, ob sie die Einrichtung außerhalb der Studie abgebrochen hätten, 
antworteten 3 von 11, dass sie abgebrochen hätten. Gründe waren einmal die 
Fehlfunktion der App (Bug) und dass sie lieber eine andere traditionelle TOTP-App genutzt 
hätten, da ihnen die Einrichtung mit Blue TOTP zu komplex war. Nur ein Proband empfand die Einrichtung mit Blue TOTP leichter als mit 
dem traditionellen Verfahren, aufgrund der ausführlichen Anleitung. Fünf meinten 
es sei schwieriger und die restlichen 5 meinten es weder leichter noch 
schwieriger. Während sich 9 von 11 Teilnehmern zutrauen würden, einer anderen 
Person bei der Einrichtung des traditionellen Verfahrens behilflich zu sein, 
waren es mit gleichen Frage analog zu Blue TOTP 8 Teilnehmer.

\paragraph*{Wahrnehmung}
\mbox{} \vspace{0.1cm} \\
Die letzte Frage zielte darauf, ob die Probanden die Einrichtung mit Blue TOTP 
als authentisch, also als festen Bestandteil der Einrichtung wahrgenommen haben 
oder ob es für sie eher aufgesetzt wirkte. Dabei gaben 4 von 11 an, dass es 
aufgesetzt wirkt. Gründe sind der ständige Wechsel zwischen App und Extension 
und dass das Design der Extension stark vom Chrome Design abweiche. Auf die 
anderen Probanden wirkte es authentisch, unter anderem weil es im Browser (als 
Extension) integriert ist und im Hintergrund läuft bzw. weil die Designs in App 
und Extension ähnlich seien. Ein Teilnehmer gab an, dass es für ihn durch 
Bluetooth sicherer wirkt.