Bei diesem Verfahren generiert der Dienstanbieter vorab einige Codes und lässt sie 
dem Nutzer zukommen (Email, Papierform, direkte Anzeige der Codes nach Einrichtung 
des Verfahrens). Diese Codes sind jeweils nur eine zufällige Zeichenabfolge, die im 
besten Fall nur einmal gültig ist. Bei der Anmeldung gibt der Nutzer nach Eingabe 
des Benutzernamens und des Passworts einen der Codes ein und ist erfolgreich 
angemeldet. Den verwendeten Code kann er dann verwerfen.
\\\\
Das Prinzip ist einfach aufgebaut (Code ablesen, eintippen und von der Liste streichen), aber es gibt einige Schwachstellen. Die Codes müssen  auf beiden 
Seiten sicher verwahrt sein. Der Anbieter kann die Codes sicher hashen, doch der 
Nutzer muss sie entweder in Papierform verwahren (im Klartext) oder in einer Datei 
(bestenfalls verschlüsselt) speichern. Außerdem sind die Codes so lange gültig, bis 
der Nutzer sie verwendet. D.h. mittels Brute-Force-Attacken, kann ein Angreifer den 
Code erraten. Aus diesem Grund sollten die Codes auch eine ausreichende Anzahl an 
Zeichen zählen. Das wiederum erfordert, dass der Nutzer bei der Eingabe des Codes 
viele Zeichen ablesen und eintippen muss. Dadurch steigt die Fehleranfälligkeit und 
der Zeitaufwand bei der Eingabe. Außerdem besteht ein Problem, wenn der Nutzer die 
Codes verliert bzw. versehentlich löscht, ohne Sicherheitskopien angelegt zu haben. \autocite[359 \psq]{Reese}
\\\\
Nach der Studie von \textcite{Reese} haben die generierten Codes bei der 
Authentisierung (nicht der Einrichtung) einen Medianwert von 80 und einen 
arithmetischen Mittelwert von $80{,}2$ in der SUS-Bewertung. Somit 
liegen sie im Vergleich zu den anderen untersuchten 2FA-Verfahren im mittleren 
Bereich. Zeitlich haben sie den größten Aufwand mit einem Median von $17{,}2~s$ und einem 
arithmetischen Mittelwert von $28~s$. Ob die Teilnehmer ihr 
benötigtes Mittel (hier die Codes) für den Zweiten-Faktor-Schritt bereits vor der 
Anmeldung bereitliegen hatten, wurde in der Studie nicht unterschieden. D.h. bei den 
gemessenen Zeiten ist stets unklar, ob und wie viel Zeit für das Beschaffen des 
Mittels benötigt wurde \autocite[364]{Reese}. Auch wurde nicht erwähnt, wie viele Zeichen jeder 
Code enthält.
