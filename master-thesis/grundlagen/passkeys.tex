Passkeys bauen auf den FIDO-Standards auf. Genau genommen basiert es 
auf dem FIDO2-Projekt \autocite{PkFIDO}. Das Ziel ist es, nicht mehr 
auf Passwörter und Benutzernamen sowie 2FA zu setzen, sondern auf eine 
passwortlose Authentisierungsmethode, die angenehmer, schneller und 
sicherer als Passwörter (und 2FA) sind. Passkeys sind 
Phishing-resistent.
\\\\
Das FIDO2-Projekt ähnelt in seinen Grundzügen dem beschriebenen 
U2F-Standard. Es hat allerdings das Ziel, die Identität eines Nutzers 
nicht anhand seines Benutzernamen und Passworts sowie ggf. einem 
zweiten Faktor zu bestimmen, sondern anhand von asymmetrischer 
Kryptographie. D.h. der Nutzer benötigt dann bei der Anmeldung nur noch 
einen privaten kryptographischen Schlüssel und der Dienst kann dessen 
Anmeldung mit dem zugehörigen öffentlichen Schlüssel verifizieren. 
Dabei kommen die zwei Protokolle WebAuthn \autocite{WebAuthnSpec} und 
CTAP2 \autocite{CTAPSpec} zum Einsatz. WebAuthn legt die Kommunikation 
zwischen Dienst (Website) und Client (Browser, Betriebsystem) fest. Der 
Dienst sendet eine Challenge an den Client, dieser identifiziert die 
Domain des Dienstes. Das CTAP2-Protkoll regelt, wie der Client die 
Challenge und Domain an den Authenticator sendet, also das Gerät, auf 
dem der private Schlüssel gespeichert ist. Der Authenticator kann dabei 
ein Smartphone, ein Security Key (z.B. YubiKey) oder selbst das Gerät 
des Clients sein (bspw. mit Windows Hello). Der private Schlüssel 
sollte immer in einem Secure Element \autocite{BSISecEl} gespeichert 
werden, einer eigenständigen physischen Komponente innerhalb des Geräts 
(Authenticators) zur sicheren Verwahrung sensibler Daten. Nun 
verifiziert der Authenticator die Domain und signiert die Challenge mit 
seinem privaten Schlüssel. Anschließend sendet er die signierte 
Challenge zurück an die Client-Instanz und diese leitet sie weiter an 
den Dienst. Der entschlüsselt die Signatur mit seinem öffentlichen 
Schlüssel und vergleicht sie mit der originalen Challenge. Nun ist der 
Client authentifiziert. \autocite{YubiFIDO2}
\\\\
Passkeys Verfahren genau nach der eben beschriebenen Prozedur. 
Allerdings war ein ursprünglicher Gedanke, dass der private Schlüssel 
nie den Authenticator verlässt. Um die Nutzung von Passkeys so angenehm 
wie möglich zu gestalten, wurde diese Beschränkung aufgehoben. Somit 
können Passkeys auf mehreren Geräten über einen Drittanbieter (z.B. 
Google) synchronisiert werden. D.h. der Nutzer muss nur auf einem Gerät 
den Passkey einrichten und kann ihn auf seine anderen Geräte durch den 
Drittanbieter übertragen lassen. So muss der Nutzer nicht für jedes 
Gerät einen Passkey bei einem Dienst einrichten, sondern nur einen 
Passkey für alle seine Geräte. Die Synchronisierung ist Ende-zu-Ende-verschlüsselt, also hat der Drittanbieter keine Einsicht auf die 
privaten Schlüssel. Es sei angemerkt, dass der Nutzer meist auf dem 
Authenticator eine Autorisierung zur Nutzung des privaten Schlüssels 
durchführen muss, z.B. durch einen Scan des Gesichts oder des 
Fingerabdrucks. D.h. nicht, dass die biometrischen Daten für die 
eigentliche Authentisierung beim Online-Dienst genutzt werden. \autocite
{PkFIDORes}
\\\\
Verwendet man ein Smartphone als Authenticator und möchte sich bei 
einem Passkey-losen Gerät (Client) authentisieren, dann muss das 
Smartphone über Bluetooth oder NFC mit dem Client (Browser, 
Betriebssystem des Computers) kommunizieren. Bei der 
Bluetooth-Verbindung vertraut man nicht auf die Sicherheitsmechanismen 
von Bluetooth Low Energy, sondern sichert die Verbindung auf 
Anwendungsebene \autocite{PkFIDORes}. Fraglich bleibt, wie man sich 
bspw. auf einem fremden Computer (ohne Bluetooth, NFC und Kamera) bei 
einem Dienst authentisiert, der bereits Passkeys voraussetzt, weil man 
es bspw. schon mit seinem Smartphone eingerichtet hat. Da Passkeys 
anscheinend eine gute Lösung des gesamten Passwort- und 
2FA-Dilemmas sind, stellt sich die Frage, wieso wir unsere Passwörter 
noch nicht durch Passkeys ersetzt haben (siehe \textcite{FidoRescue} und 
\textcite{lassakaren}).