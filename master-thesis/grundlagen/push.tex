Push-Benachrichtigungen sind Nachrichten, die beim Empfang auf dem Smartphone 
angezeigt werden. Meist kann der Nutzer mit ihnen interagieren oder direkt die 
zugehörige App öffnen. Einige Anbieter setzen die 2FA mit Hilfe dieser 
Push-Benachrichtigung um. D.h. der zweite Faktor ist hier die direkte Nachfrage vom Dienst beim Nutzer, ob sich dieser gerade anmeldet. In der Regel muss der Nutzer eine App des Anbieters auf 
seinem Smartphone installieren und sich dort mit seinen Zugangsdaten anmelden. Es 
gibt auch Apps wie Authy OneTouch \autocite{authy}, die als Drittanbieter für die 
Authentisierung agieren. Somit kann man Authentisierungen für verschiedene Dienste, 
die Authy OneTouch unterstützen, mit einer App bestätigen. Immer wenn der Nutzer 
sich dann bspw. im Browser am Computer beim Dienst anmeldet, bekommt er eine 
Push-Benachrichtigung. Diese enthält einen Text wie \glqq Melden Sie sich gerade an?\grqq{} 
oder \glqq Bestätigen Sie Ihre Anmeldung\grqq{} und ggf. schon zwei Buttons zum Bestätigen oder 
Ablehnen des Anmeldungsversuchs. Bestätigt der Nutzer die Anmeldung, dann hat er 
sich erfolgreich im Browser beim Dienst angemeldet.
\\\\
Dazu benötigt das Smartphone einen Internetzugang. Die Verbindung zwischen 
Smartphone und Anbieter muss selbstverständlich sicher sein. Trotzdem gibt es einige 
Schwachstellen. Falls der Angreifer weiß, zu welchem Zeitpunkt der Nutzer sich 
anmelden wird, kann er sich selbst kurz vorher anmelden. Dadurch sieht der Nutzer 
zwei Anmeldeversuche, die bestätigt werden können. Ist er in dem Moment nicht 
achtsam und bestätigt einen Anmeldungsversuch, dann ist es möglich, dass er nicht 
seine Anmeldung, sondern die des Angreifers bestätigt. \autocite[16]{Hess}
\\\\
Besonders von Interesse ist die sogenannte MFA-Fatigue \autocite{sosafe}. Dabei hat der Angreifer 
Benutzername und Passwort des Opfers bereits erlangt und versucht nun immer wieder, 
sich beim Dienst anzumelden. Das Opfer ist irgendwann genervt von all den 
Anmeldungsanfragen, sodass es absichtlich die Anmeldung bestätigt und dem Angreifer 
Zugriff gewährt. Natürlich ist es auch möglich, dass das Opfer die Anmeldung des 
Angreifers bestätigt, weil es unachtsam ist und dann aus Gewohnheit die Nachricht 
bestätigt. In diesem Fall musste der Angreifer nicht zwingend viele 
Anmeldungsversuche unternehmen. Ein brisantes Beispiel ist der erfolgreiche 
Cyberangriff auf Uber im Jahr 2022, bei dem MFA-Fatigue von Bedeutung war \autocite{Huckle}.
\\\\
In der Studie von \textcite{Reese} haben die Teilnehmer der Push-Studiengruppe Authy 
OneTouch benutzt und nicht eine eigens für die Studie erstellte App. Dennoch sollte 
es in der Handhabung und in der benötigten Zeit keine nennenswerten Unterschiede 
geben. Die SUS-Bewertung liegt für den Median bei $81{,}3$ und für den arithmetischen 
Mittelwert bei 81. Erstaunlich ist, dass die vorab generierten 
Codes eine ähnliche Bewertung von rund 80 Punkten je Größe aufweisen. Hinsichtlich 
der benötigten Zeit waren es im Median $11{,}8~s$ und im Mittelwert $16{,}1~s$. Somit wäre das Push-Verfahren das zweitschnellste nach dem untersuchten 
U2F-Verfahren.
