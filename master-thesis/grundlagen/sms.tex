Ein bekanntes Verfahren sind Einmalpasswörter (meist sechsstellige Zahlen), die nach 
Eingabe von Nutzername und Passwort per SMS an das Smartphone des Nutzers gesendet 
werden. Auch die Übertragung des Einmalpassworts per E-Mail ist üblich, soll hier 
aber nicht näher betrachtet werden. Bzgl. der Nutzerfreundlichkeit teilen SMS und 
E-Mail sich einige Eigenschaften. Der Nutzer liest den empfangenen Code ab und gibt 
ihn dann auf der Login-Seite ein. Nach der Verwendung verliert das Einmalpasswort 
seine Gültigkeit. Diese sollte im Idealfall nur für wenige Minuten gültig sein.
\\\\
Die Methode wird oft genutzt und hat wie die anderen smartphone-basierten Verfahren 
den Vorteil, dass kein extra Gegenstand benötigt wird. Jedoch hat das SMS-Verfahren 
ähnliche Kritikpunkte wie die generierten Codes. Auch hier muss der Code abgelesen 
und eingetippt werden, wobei Zeichen fehlerhaft übernommen werden könnten. Außerdem 
ist auch hier der Verlust des Geräts ein Problem. Dafür könnte man die vorab 
generierten Codes als Backup-Lösung nutzen, sofern der Dienst dies unterstützt. Aber 
auch dies öffnet, wie bereits erwähnt, Angriffsmöglichkeiten.
\\\\
Ein weiteres Problem ist das SIM-Swapping. Hierbei kontaktiert der Angreifer den 
Kundenservice des Mobilfunkanbieters des Nutzers. Mit einem plausiblen Grund (z.B. 
Verlust des Smartphones) gelingt es dem Angreifer, den Mobilfunk-Account des Nutzers 
einer neuen SIM-Karte zuzuweisen, die ihm der Mobilfunkanbieter dann zusendet. Von 
nun an empfängt der Angreifer die SMS-Nachrichten, die eigentlich an den Nutzer 
gerichtet sind. \autocite[51 \psq]{Jover}
\\\\
Jovers Zusammenfassung zufolge gibt es in den Netzwerken des Global System for 
Mobile Communications (GSM) sowie Long-term Evolution (LTE) Schwachstellen, die es 
dem Angreifer ebenfalls ermöglichen, den SMS-Verkehr des Nutzers auf das Gerät des 
Angreifers umzuleiten. Voraussetzung sei, dass der Angreifer Zugriff auf einen 
Entry-Point des \glqq Signaling System 7\grqq{}-Netzwerks benötigt. \autocite[51]{Jover}
\\\\
Auch der später in Kap. \ref{sec: phishing} (S. \pageref{sec: phishing}) beschriebene Phishing-Angriff ist eine Schwachstelle des 
SMS-Verfahrens (gilt auch für die E-Mail-Variante). Aufgrund dieser sicherheitstechnischen Schwachstellen des SMS wurde das smsTAN-Verfahren in der EU abgeschafft \autocite{BSIsmsTan}.
\\\\
Nach der Studie von \textcite{Reese} wurde die SMS-Methode bei der Authentisierung 
mit einer SUS-Punktzahl von 75 (Median und arithmetischer Mittelwert) bewertet. Somit ist sie zusammen mit der U2F-Methode die am schlechtesten 
bewertete von den fünf untersuchten Methoden. Bzgl. der Zeitmessung sind es $16{,}6~s$ 
beim Median, wobei SMS mit den generierten Codes am langsamsten wäre. Im arithmetischen Mittel sind es $18{,}5~s$, womit SMS nahe dem Mittelwert der 
Push-Methode ($16{,}1~s$) kommt und eher zu den schnelleren Methoden zählen würde.
