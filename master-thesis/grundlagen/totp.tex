Anstatt das Einmalpasswort per SMS oder E-Mail zu empfangen, gibt es die 
Möglichkeit, es selbst zu berechnen. Somit benötigt man nicht 
zwingend eine Verbindung zum Mobilfunknetz bzw. zum Internet mit dem Gerät, welches 
das Einmalpasswort berechnet. Man kann Einmalpasswörter mit dem Verfahren HMAC-based 
One-time Password (HOTP) oder dem Verfahren Time-based One-time Password (TOTP) 
erstellen. Sie sind durch den Standard RFC 6238 definiert \autocite{rfc6238}. Der Unterschied ist 
nur, dass beim HTOP-Verfahren ein Zählwert als Eingabe genutzt wird, der entweder 
vor der Eingabe bekannt gegeben wird oder nach jeder Nutzung inkrementiert wird, und 
beim TOTP-Verfahren dagegen ein stetig steigender Wert, nämlich die Unixzeit 
unterteilt in $n$-Sekunden-Schritte (i.d.R. $n = 30$). Die Unixzeit ist eine Ganzzahl, die die 
vergangene Zeit seit dem 1.1.1970 um 00:00 Uhr in Sekunden repräsentiert. Im 
Folgenden werden nur zeitbasierte Einmalpasswörter (abgekürzt mit TOTP) betrachtet, 
die von Smartphones bzw. den entsprechenden Apps berechnet werden.
\\\\
Bei der Einrichtung dieses Verfahrens zeigt der Dienstanbieter dem Nutzer meist 
einen Quick Response (QR) Code auf seiner Website. Der QR-Code enthält ein Geheimnis, also eine Zeichenfolge, die auch als Shared Secret bezeichnet wird. Der Nutzer scannt den QR-Code mit einer entsprechenden App auf 
seinem Smartphone. Aus dem Geheimnis und der aktuellen Unixzeit berechnet die App 
das TOTP, eine meist sechsstellige Zahl. Meldet sich der Nutzer dann erneut beim 
Dienst an, wird er nach Eingabe des Benutzernamens und des Passworts nach dem TOTP 
gefragt. Der Nutzer öffnet seine App, liest das TOTP ab und tippt es in den Browser 
ein. So kann eine solche App mehrere Geheimnisse für jeden Dienst, der das Verfahren 
unterstützt, speichern und entsprechend die TOTPs berechnen.
\\\\
Allerdings bringt auch dieses Verfahren einige Probleme mit sich. Zum einen muss der 
Nutzer für die Authentisierung erst die App auf seinem Smartphone suchen, öffnen und 
dann den Account auswählen, bei dem er das TOTP benötigt. Zum anderen muss der 
Nutzer das TOTP wie beim SMS-Verfahren ablesen und eintippen (fehleranfällig). Hinzu 
kommt, dass ein TOTP in der App maximal nur für 30~s angezeigt wird und danach ein 
neues. Wenn eine Person körperlich oder kognitiv nur langsam agieren kann, dann 
könnten die 30~s nicht ausreichend sein. Dadurch kann sich eine solche Person beim 
Abtippen des TOTPs unter Druck gesetzt fühlen und anfälliger für Tippfehler werden.
Des Weiteren gibt es Fälle, in denen das aktuelle TOTP nur noch für kurze Zeit (z.B. 5s) auf dem Display 
verweilt, bis das neue angezeigt wird. Nutzer warten dann, bis das neue TOTP 
angezeigt wird, weil die verfügbare Zeit nicht ausreichend gewesen wäre, um es 
abzulesen und im Browser einzugeben.
\\\\
Die Schwachstellen, die es beim SMS-Verfahren gibt, sind hier kein Problem, bis auf 
das später in Kap. \ref{sec: phishing} (S. \pageref{sec: phishing}) beschriebene Phishing. Das Geheimnis benötigt eine ausreichende 
Länge, sonst kann es leicht erraten werden. RFC 6238 schreibt keine Mindestlänge 
vor, aber 32~Byte sind üblich. Ein eher leicht zu lösendes Problem ist, dass die Uhr 
des Smartphones und des Diensteanbieters synchronisiert sind. Optimalerweise akzeptiert der Dienst nicht nur das aktuelle TOTP, sondern auch das TOTP im Zeitfenster davor und danach. Somit ist ein TOTP effektiv nicht nur 30~s sondern 90~s gültig. Auf diese Weise werden geringe Abweichungen in den Uhren der beiden Parteien ausgeglichen \autocite[7]{rfc6238}.
\\\\
Den Ergebnissen aus der Studie von \textcite{Reese} zufolge erreicht die 
Authentisierung mit den TOTPs eine SUS-Bewertung von $88{,}8$ im Median und $83{,}1$ im 
arithmetischen Mittelwert. In beiden Größen hat TOTP unter den 
untersuchten 2FA-Verfahren die beste Bewertung (Passwort ausgenommen), gefolgt vom 
Push-Verfahren mit rund 81 Punkten in beiden Größen. Bzgl. der Zeitmessung schneiden 
die TOTPs mit einem Median von $15{,}1~s$ und einem arithmetischen Mittelwert von $23{,}9~s$ 
eher mittelmäßig bis schlecht ab.
