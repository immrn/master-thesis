Der Universal Second Factor (U2F) ist ein Standard zur Zwei-Faktor-Authentisierung, 
der von der Fast Identity Online (FIDO) Alliance entwickelt wurde. Der Standard 
bietet den besten Schutz auf externer Hardware wie einem USB-Gerät (z.B. der 
YubiKey), kann aber auch als reine Softwarelösung umgesetzt werden \autocite[4]{u2f}. Ist 
das U2F-Gerät eingerichtet, meldet sich der Nutzer beim Dienstanbieter mit 
Benutzername und Passwort an. Danach muss er lediglich eine Interaktion mit dem 
U2F-Gerät tätigen, damit dieses ihn vollständig beim Dienstanbieter authentisiert. 
Die Interaktion kann auf verschiedene Weisen umgesetzt werden. Die einfachste Form 
ist das Drücken eines Knopfes am U2F-Gerät, aber auch biometrische Eingaben wie der 
Scan der Iris oder des Fingerabdrucks sind möglich. Evtl. kann es störend sein, dass 
der Nutzer bei dieser Methode einen zusätzlichen Gegenstand mit sich führen muss.
\\\\
Bei der Einrichtung des Verfahrens stellt die Website innerhalb des Browsers mit 
einer Javascript-Funktion eine Anfrage an das U2F-Gerät. Nun interagiert der Nutzer 
mit seinem U2F-Gerät (z.B. Drücken eines Knopfes), um die Anfrage zu bestätigen. 
Vereinfacht erklärt ist der weitere Verlauf wie folgt. Das U2F-Gerät erstellt dann 
ein asymmetrisches Schlüsselpaar. Dieses verwahrt es zusammen mit Informationen wie 
der Domain des Dienstanbieters in seinem eigenen Speicher. Dann übergibt das 
U2F-Gerät den öffentlichen Schlüssel mit einigen anderen Informationen an den 
Dienstanbieter (im Hintergrund über den Browser). Pro Dienstanbieter wird ein neues 
Schlüsselpaar erzeugt. Beim eigentlichen Authentisierungsvorgang erhält das 
U2F-Gerät Informationen wie die Domain, die vom Browser selbst geprüft wurden, und 
identifiziert so den benötigten privaten Schlüssel. Das U2F-Gerät signiert einige 
der erhaltenen Informationen und sendet sie (über den Browser) zurück an den 
Dienstanbieter. Dieser verifiziert die Signatur dann mit dem zugehörigen 
öffentlichen Schlüssel. Man kann außerdem mehrere U2F-Geräte für denselben Dienst 
nutzen und außerdem können mehrere U2F-Geräte an ein Gerät angeschlossen werden. 
Darüber hinaus gibt es auch Lösungen mit Near Field Communication (NFC), um das 
U2F-Gerät an einem Smartphone zu verwenden, und mit Bluetooth. \autocite{u2f}
\\\\
Durch seine Sicherheitsmechanismen hat U2F bisher keine bekannten 
Sicherheitslücken. In der Praxis kann es durchaus vorkommen, dass in der gesamten 
Kette des Protokolls eine Implementierung fehlerhaft vorgenommen wurde. 
Beispielsweise kann eine Lücke im Browser oder im U2F-Gerät entstehen. So konnten 
Forscher im Jahr 2021 mit einer Side-Channel-Attack die Informationen eines Google 
Titan Security Key kopieren \autocite{Roche}. Dem sei anzumerken, dass eine solcher Angriff 
komplex ist und nur mit hochpreisigen Geräten (ca. 12.000 US-Dollar) vorgenommen 
werden kann. Weitere Angriffe sind bspw. bei der Einrichtung möglich, wenn die 
Website oder eine installierte Browser-Extension bösartig sind \autocite{Yadav}. Bei Verlust 
eines U2F-Gerätes sollte man ein Backup-2FA-Verfahren wie TOTP oder Push 
eingerichtet haben.
\\\\
In der Studie von \textcite{Reese} haben die Teilnehmer der U2F-Gruppe 
sich mittels YubiKeys authentisiert. Die SUS-Bewertung liegt im Median bei 75 und im 
Mittelwert bei $73{,}1$ Punkten. D.h. zusammen mit dem SMS-Verfahren 
(75 Punkte im Median und Mittelwert) ist es das am schlechtesten bewertete Verfahren 
der Studie. \textcite{Reese} vermuten, dass einige Teilnehmer den Yubikey verkehrt herum 
angeschlossen haben (schmaler USB-Stecker). Dagegen ist U2F zeitlich betrachtet mit 
$9{,}1~s$ im Median und $13{,}0~s$ im Mittelwert das schnellste Verfahren der Studie.
