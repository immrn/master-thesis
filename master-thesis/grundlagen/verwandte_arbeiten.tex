Die Problemstellung, dass bis auf FIDO-basierte Verfahren die 2FA anfällig für Phishing ist, ist 
nicht neu und dennoch existiert sie. Nicht jeder Nutzer möchte bspw. einen Security Key, da er 
ihn die meiste Zeit bei sich tragen muss. Genauso gibt es immer noch viele Websites, die 
Phishing-anfällige Verfahren wie TOTP oder Push nutzen und teilweise nicht einmal die 
Phishing-resistenten Alternativen anbieten. Man handelt nach der Devise \glqq Ein lückenhafter 
Schutz ist besser als gar kein Schutz\grqq{}, und im ersten Augenblick mag diese Behauptung 
stimmen. Aber in Wahrheit verschiebt sich die Angriffsfläche nur: weg von Brute-Force- und 
Wörterbuch-Angriffen (usw.) hin zum Real-time Phishing.
\\\\
2FA-PP steht für 2FA Phishing Prevention und wird in der Arbeit von \textcite{2FAPP} vorgestellt. Es ist ein 
Phishing-resistentes 2FA-Verfahren, bei dem der Browser als zu vertrauende Partei angesehen wird. 
Es kann unter anderem mit One-time Passwords (OTP) kombiniert werden. Es nutzt die Web Bluetooth 
API, damit die Website über den Browser mit dem Smartphone kommuniziert. Nachdem der Nutzer 
seinen Benutzernamen und sein Passwort eingegeben hat, verifiziert die Website diese und sendet 
einen verschlüsselten Javascript-Code an den Browser. Der Browser verlangt vom Smartphone den 
Schlüssel. Das Smartphone sendet den Schlüssel und beginnt eine Zeitmessung. Der Browser erhält 
den Schlüssel, entschlüsselt den Javascript-Code und führt ihn aus. Er sendet das Ergebnis des 
ausgeführten Codes zurück ans Smartphone, das wiederum die Zeitmessung stoppt. Ist das erhaltene 
Ergebnis gültig und die Zeitmessung kleiner als ein bestimmter Grenzwert, gibt das Smartphone der 
Website direkt die Bestätigung, dass der zweite Faktor verifiziert wurde oder das Smartphone 
zeigt bspw. das OTP an, das der Nutzer dann in die Website eintippt. Die Idee scheint 
vielversprechend, da der Nutzer im Idealfall nur Benutzername und Passwort eingeben muss, alles 
2FA-spezifische geschieht im Hintergrund. Problematisch könnte hier allerdings der Fakt sein, 
dass das Smartphone einen Schlüssel an den Browser sendet. Das Senden von privaten Schlüsseln ist eine 
schlechte Praxis \autocite{DIGI}. Außerdem ist auch das Stoppen der Zeit fragwürdig. Der 
Mechanismus basiert darauf, dass der Javascript-Code von einem Angreifer nicht in der kurzen Zeit 
entschlüsselt werden und ausgeführt werden kann. Sollte der Angreifer den Schlüssel erlangen, ist 
dieser Mechanismus hinfällig. Die Idee von 2FA-PP setzt voraus, dass Websites das Konzept von 
2FA-PP unterstützen.
\\\\
Ein ungewöhnlicher Ansatz 2FA nutzerfreundlicher zu gestalten ist 2D-2FA \autocite{2D2FA}. Dabei 
steht 2D für zweidimensional. Die Website zeigt eine Kennung, genauer ein Muster. Das Muster 
funktioniert genauso wie das Wischmuster von Android zum Entsperren des Smartphones. Der Nutzer 
sieht also dieses Muster auf der Website, gibt es in die Smartphone-App ein und dieses generiert 
dann eine PIN, die über das Internet an den Server der Website gesendet wird. Die Idee ist, dass 
der Nutzer nur noch ein Muster auf dem Smartphone eingeben muss für den zweiten Faktor. 
Sicherheitstechnisch gesehen, bringt dieses Konzept allerdings keinen Mehrwert. Im Vergleich zu 
TOTPs tauscht man aus Nutzersicht nur das Eingabegerät. Beim TOTP-Verfahren gibt der Nutzer das 
TOTP in die Website ein, hier gibt der Nutzer prinzipiell ein zufällig generiertes Muster in sein 
Smartphone ein. Das Verfahren schützt genauso wenig vor Phishing wie TOTP. Letztlich ist auch der 
Unterschied nicht revolutionär, ob man als Nutzer nun eine Zahl oder ein Muster ablesen und 
eingeben muss.
\\\\
Ein ebenfalls ungewöhnlicher Ansatz, um 2FA Phishing-resistent zu gestalten und dies mit dem 
Smartphone des Nutzers zu lösen, ist PhotoAuth \autocite{PhotoAuth}. Die Idee ist, die Domain als 
zweiten Faktor zu nutzen. Dabei meldet sich der Nutzer bei einer Website mit Benutzername und 
Passwort an. Die Website sendet dann bspw. per SMS (oder einen anderen Weg) einen Link an das 
Smartphone des Nutzers. Der Nutzer öffnet den Link in einem Browser seines Smartphones. Dort muss 
er der Website erlauben, ein Foto mit seiner Smartphone-Kamera aufzunehmen. Er fotografiert die 
Adresszeile seines PC-Browsers ab, zumindest so, dass die Domain scharf im Bild ist. Dieses Bild 
wird vom Smartphone-Browser dann hochgeladen zum eigentlichen Server, wo sich der Nutzer 
authentisieren möchte. Mithilfe von Optical Character Recognition wird die Domain erkannt und 
geprüft, ob es sich um die echte Domain und nicht um eine Phishing-Domain handelt. D.h. man 
umgeht das Real-time Phishing. Denn bei einem Phishing-Angriff würde der Nutzer unwissentlich 
Benutzername und Passwort an der Angreifer geben, der es an die echte Website weiterleitet. 
Allerdings ist der Kommunikationsweg für den zweiten Faktor ein ganz anderer als der für 
Benutzername und Passwort. Das Foto wird vom Smartphone des Nutzers direkt zum Server der echten 
Website gesendet und dieser würde auf dem Bild feststellen, dass die Domain nicht seiner Domain 
entspricht. Typosquatting könnte je nach Schriftart des Browsers eine ernsthafte Schwachstelle 
sein (siehe Abb. \ref{fig: gitlab}). Man könnte argumentieren, dass der Angreifer einfach selbst 
ein Bild an den Server der echten Website sendet, auf dem die echte Domain zu sehen ist. 
Allerdings kennt er nicht den Link, den der Nutzer per SMS erhalten hat. Denn dieser Link ist für 
jede Anmeldung individuell. Sicherheitstechnisch scheint es bis auf das Typosquatting oder andere 
Möglichkeiten, die Optical Character Recognition zu täuschen, keine Schwachstellen zu geben. 
Hinsichtlich der Nutzerfreundlichkeit ist das Verfahren offensichtlich untauglich, um 
nennenswerte Fortschritte zu erzielen.
\\\\
Zuletzt sei noch 
1Password\footnote{\href{https://support.1password.com/one-time-passwords/}{https://support.1password.com/one-time-passwords/}} 
erwähnt. Es hat das Ziel TOTPs zu vereinfachen. Die Idee ist es, die Geheimnisse online zu 
speichern (und in der App sowie in der Browser-Extension). Nutzt man für die Anmeldung bei einer 
Website die 1Password Browser-Extension, dann erkennt diese automatisch das TOTP-Eingabeelement, 
generiert das TOTP und trägt es automatisch ein. Aus Sicht des Nutzers ist es  angenehm, 
aber hier speichert man seine Geheimnisse remote bei 1Password. D.h. sollte 1Password Opfer eines 
erfolgreichen Angriffs werden, der die Geheimnisse entwendet, ist der zweite Faktor 
kompromittiert. Es sei noch angemerkt, dass 1Password nur gegen monatliche Zahlungen nutzbar ist.