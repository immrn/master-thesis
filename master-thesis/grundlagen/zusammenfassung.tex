Fasst man die Ergebnisse der Studie von Reese et al. \autocite{Reese} zusammen, ist U2F die 
schnellste Methode für die Zwei-Faktor-Authentisierung. Und trotzdem erreicht U2F 
zusammen mit SMS die niedrigste SUS-Punktzahl. Gegenteilig dazu verhalten sich die 
Werte zu den TOTPs. Deren Zeitaufwand ist verhältnismäßig zu den anderen Verfahren 
mittel bis hoch, aber die SUS-Bewertung ist die höchste (Passwörter ausgenommen). 
Das zweitschnellste Verfahren sind die Push-Benachrichtigungen, allerdings soll es 
vorgekommen sein, dass Teilnehmer die Push-Nachricht nicht empfingen und die 
entsprechende App selbst öffnen mussten, um anschließend den Zugang zum Dienst zu 
bestätigen. \autocite[10]{Reese}
\\\\
Jedoch sollte man den SUS-Bewertungen nicht allzu viel Bedeutung zuschreiben. Eine 
andere Studie von \textcite{Acemyan} hat gezeigt, dass Push die höchste 
SUS-Punktzahl mit ca. 87 (Median) erreichte, während TOTP (ebenfalls wie bei der Studie von \textcite{Reese} mit dem Google Authenticator) nur ca. 70 Punkte erreichte. U2F erreichte ca 
77 Punkte. D.h. die SUS-Bewertung von Reese et al. widerspricht der Bewertung von 
Acemyan et al. Zu der Studie von Acemyan et al. sei zu sagen, dass es 20 Teilnehmer 
waren, die innerhalb einer Veranstaltung vier verschiedene 2FA-Verfahren genutzt 
haben. D.h. sie haben das Verfahren nur einmal getestet und dann den jeweiligen 
SUS-Fragebogen ausgefüllt. \autocite{Acemyan}
\\\\
Betrachtet man die Verfahren dahingehend, welche Interaktionen der Nutzer vornehmen 
muss, dann sind U2F Security Keys sehr gering im Aufwand. Man schließt den Security 
Key an und drückt einen Knopf, um sich zu authentisieren. Allerdings benötigt man 
einen neuen Gegenstand für die 2FA. Dagegen besitzen die meisten Personen, 
die 2FA nutzen, ein Smartphone. D.h. sie benötigen keinen neuen Gegenstand. Daher 
sind Push und TOTP attraktiv. TOTP benötigt im Vergleich zu Push 
weitere Interaktionen. Man muss das Smartphone entsperren, die App finden, öffnen, 
den Eintrag für den entsprechenden Dienst finden und dann die sechs Ziffern ablesen 
und im Browser eintippen. Bei Push muss man lediglich das Smartphone entsperren, in 
die Benachrichtigungen schauen und bei der entsprechenden Benachrichtigung auf 
\glqq Zustimmen\grqq{} oder \glqq Ablehnen\grqq{} tippen. Diesbezüglich ist SMS eine Mischung aus beiden 
Verfahren (Benachrichtigung und Abtippen). So spiegeln auch die Zeiten von Push und 
U2F ihren eher niedrigen Interaktionsaufwand wider.
\\\\
Den Aspekt der Sicherheit sollte man allerdings mehr Bedeutung zuschreiben, denn 
mehr Sicherheit ist der Grund, wieso man 2FA betreibt. Die vorab generierten Codes 
sind das unsicherste Verfahren, da sie nur ein weiteres Passwort sind, 
dass erraten werden kann. Schon etwas mehr Sicherheit bietet SMS, allerdings hat es 
Schwächen seitens des Mobilfunks, die bspw. bei den TOTPs entfallen. Beide Verfahren 
(SMS und TOTP) sind genauso wie Push-Benachrichtigungen nicht Phishing-resistent. 
Bei Push kommt die Gefahr der Gewohnheit, jede Authentisierungsanfrage zu 
bestätigen, bzw. MFA-Fatigue hinzu. Das sicherste Verfahren ist U2F, solang der 
Computer bzw. Browser frei von entsprechender bösartiger Software ist.
