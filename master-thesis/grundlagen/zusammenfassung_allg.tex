Neben den Grundlagen bekannter 2FA-Verfahren wurden deren Probleme erörtert, wie die Anfälligkeit 
gegen Phishing oder schlechte Gebrauchstauglichkeit. Auch wurden Forschungsarbeiten zum Thema 
vorgestellt, die versuchen, 2FA neu zu denken. Passkeys stellen die gesamte 
Zwei-Faktor-Authentisierung in Frage und scheinen eine gute Idee für zukünftige Systeme zu sein. 
Sie könnten deutlich mehr verbreitet sein. Allerdings sind Benutzernamen und Passwörter tief 
verankert und nicht von heute auf morgen wegzudenken. Genauso wenig kann man jeden Webdienst und 
jeden Nutzer dazu zwingen, sich auf \glqq das eine Verfahren\grqq{} zu beschränken.
\\\\
Die im weiteren Verlauf vorgestellte Verbesserung von TOTPs unterscheidet sich dahingehend von 
den bestehenden Systemen und entwickelten Ansätzen, dass sie nahe am TOTP-Verfahren bleibt. 
Letztlich geht es darum, das TOTP automatisch und möglichst mit wenig Nutzerinteraktionen zur 
Website zu transportieren. Dabei soll ein lokales Bluetooth-Netzwerk verwendet werden, ohne dass 
Geheimnisse bei einem Drittanbieter gespeichert werden. Das TOTP bleibt immer noch Bestandteil, 
doch nur im Hintergrund, und kann für eine Fallback-Option nützlich sein. Außerdem hat das später 
vorgestellte Konzept einen Phishing-Schutz für den zweiten Faktor.