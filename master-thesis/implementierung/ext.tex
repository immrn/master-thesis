Bei der Einrichtung und der Authentisierung muss der Browser dem Konzept zufolge 
den Benutzernamen und die Domain an die App senden. Außerdem muss er beim Login das 
HTML-Eingabeelement erkennen und das TOTP von der App verlangen. Also ist der 
Browser die Partei, der man vertrauen muss. Es muss also bei der Umsetzung der 
Browser-Partei bedacht werden, dass sie potentiell nicht von einer anderen Partei 
negativ beeinflusst werden kann.
\\\\
Nun benötigt man zunächst einen Browser, der Bluetooth unterstützt. 
Chromium\footnote{\href{https://www.chromium.org/chromium-projects/}{https://www.chromium.org/chromium-projects/}} 
ist eine Open Source und beinhaltet die sogenannte Web Bluetooth API. Aber wie kann 
man den Browser die nötigen Abläufe durchführen lassen? Man könnte von Chromium 
eine eigene Instanz erstellen (Fork) und die Software für unsere Browser-Rolle 
integrieren. Für das Konzept wäre das die optimale Option, da man den Browser 
vollständig kontrolliert. Für eine Lösung, die im Alltag funktionieren soll, würde 
es bedeuten, dass der Nutzer einen neuen Browser benötigt, nur um unser neues 
TOTP-Verfahren zu nutzen. Also unpassend. Stattdessen entschied man sich dafür eine 
Extension für den Chrome Browser zu implementieren, die ebenfalls Zugang zur Web 
Bluetooth API hat. Allerdings musste man feststellen, dass die API nicht nur der 
Extension die Fähigkeit verleiht, den Bluetooth-Adapter des Computers zu nutzen, 
sondern auch der Website, auf der man sich gerade befindet. D.h. immer, wenn die 
Extension sich mit dem Smartphone verbinden soll, würde das nur funktionieren, wenn 
der Nutzer mit dem aktuellen Browser Tab eine Website besucht, und es würde auch 
immer die Website Informationen an das Smartphone senden können. Also hat man einen 
Weg gesucht, um die Bluetooth-Funktionalität aus der Extension auszugliedern und 
über eine proprietäre Anwendung für den Computer zu lösen. Dazu entschied man sich 
für Electron\footnote{\href{https://www.electronjs.org/de/}{https://www.electronjs.org/de/}}, 
da es gut dokumentiert ist und sich ebenfalls Chromium bedient. Also hat man auch 
hier wieder einen Bluetooth-Zugang durch Web Bluetooth. Web Bluetooth ist letztlich 
Bluetooth Low Energy, aber hat das Problem, dass es nur als Central Device, also 
als Client funktioniert. D.h. unsere PC-Anwendung bzw. die Extension ist die 
Partei, die nach anderen Bluetooth-Geräten scannt. Das widerspricht der 
verbreiteten Handhabung, dass das Smartphone häufig als Central Device 
agiert und nicht als Peripheral Device. Der gute Punkt ist, dass bei Bluetooth Low 
Energy die Peripherals wenig Energie benötigen, um permanent verfügbar zu sein. 
Also muss unser Smartphone das Peripheral sein und die Extension bzw. ihre 
zugehörige Electron-Anwendung das Central.
\\\\
Die Electron-Anwendung wurde nur für Windows entwickelt, sollte aber leicht für 
andere Betriebssysteme gebaut werden können. Sie läuft im Hintergrund und ist im 
System Tray, also dem Bereich rechts unten in der Taskleiste mit den kleinen 
Symbolen zu sehen. Man muss aus Sicht des Nutzers die Anwendung einmal installieren 
und dann nie wieder beachten. Sie wird automatisch beim Start des Systems 
ausgeführt. Allerdings benötigt sie aufgrund der Architektur von Electron bis zu 
$50$ MB Arbeitsspeicher. Außerdem scheint es einen Bug zu geben, der die Anwendung mehrfach 
instanziiert. Rechenleistung benötigt sie nahezu keine, da sie nur selten 
kleine Bluetooth-Pakte sendet bzw. empfängt und diese mit der Extension 
austauscht. Diese Kommunikation zwischen der Extension und der Hintergrundanwendung 
wurde mit WebSockets\footnote{\href{https://developer.mozilla.org/en-US/docs/Web/API/WebSocket?retiredLocale=de}{https://developer.mozilla.org/en-US/docs/Web/API/WebSocket?retiredLocale=de}} 
gelöst. Allerdings ist die Verbindung nicht sicher, d.h. jede Anwendung auf dem 
Computer kann die ausgetauschten Informationen auf dem Port mitlesen oder selbst 
welche senden. In Zukunft müsste auch diese Verbindung sicher sein oder das gesamte 
Konzept anders gelöst werden, sodass für die Rolle des Browsers aus dem Konzept 
nicht zwei Programme benötigt werden.
\\\\
Ein Überblick zu allen wichtigen Screens der Extension ist in Anhang \ref{anh: blue totp ext screens} 
zu sehen. Es wurde sich für ein dunkles Design entschieden, das auf zwei 
Hintergrundfarben (dunkelgrau und ein etwas helleres grau), einer Sekundärfarbe 
(leicht gräuliches weiß für einen weichen Kontrast) für Schrift und Konturen und 
einer Primärfarbe (blau) basiert. Einige Buttons haben die Primärfarbe erhalten, 
damit sie dem Nutzer direkt auffallen. Dadurch werden in den Screens der Anleitung 
die Buttons hervorgehoben, die zum direkten Abschluss der Einrichtung führen. Die 
Intention der Anleitung ist es, Nutzer an die Hand zu nehmen, die wenig Erfahrung 
mit TOTPs haben und darüber, wie man sie einrichtet. Im Nachhinein betrachtet, ist 
diese Führung irreführend für Nutzer, die schon den QR-Code bei der 2FA-Einrichtung 
sehen und einfach nur den QR-Code scannen wollen. Im Fall, dass die 
Hintergrundanwendung nicht läuft, zeigt die Extension Anweisungen, wie man die 
Hintergrundanwendung wieder startet.