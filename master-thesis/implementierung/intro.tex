Die Implementierung setzt das in Kap. \ref{sec: funktionsweise konzept} 
vorgestellte Konzept als Prototyp um. Damit man nicht darauf bauen muss, 
dass Browser-Hersteller das Konzept in Zukunft als neuen Standard integrieren, wird 
der Prototyp als eine Lösung entwickelt, die gegenwärtig funktioniert.
Für die Rolle 
des Browsers wurde eine Chrome Extension mit einer Hintergrundanwendung für 
Windows entwickelt. Durch die Hintergrundanwendung erhält die Extension mehr 
Freiheiten bzgl. der Bluetooth-Funktionalität. Die App basiert auf der 
Authenticator-App 
\textit{FreeOTP+}\footnote{\href{https://github.com/helloworld1/FreeOTPPlus}{https://github.com/helloworld1/FreeOTPPlus}} für Android.
\\\\
Die Prototyp erhält den Namen \glqq Blue TOTP\grqq{} und wird in einem 
Repository unter folgendem Link verwaltet: \textbf{\href{https://github.com/immrn/master-thesis}{https://github.com/immrn/master-thesis}}
