Zu Beginn der Arbeit wird ein Überblick über die 
Zwei-Faktor-Authentisierung, deren gängige Verfahren sowie 
weiterführende Technologien gegeben. Dann wird ein Konzept zur 
Verbesserung von Time-based One-time Passwords für mehr 
Benutzbarkeit und Schutz gegen Phishing vorgestellt. Dabei wird 
auch der Forschungsansatz mit seinen Forschungsfragen und 
Hypothesen definiert. Danach wird mit der Implementierung des 
vorgestellten Konzepts in Form eines Prototyps fortgefahren. 
Anschließend wird die Studie vorgestellt, die der Untersuchung 
des Prototyps bezüglich seiner Nutzerfreundlichkeit, 
Nutzererfahrung und Task Completion Time dient. Am Ende werden 
die Studienergebnisse präsentiert, diskutiert und ein Ausblick 
gegeben.