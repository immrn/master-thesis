Oft besitzen Personen schwache Passwörter oder verwenden das gleiche 
Passwort für mehrere Dienste, da sich so die Zugangsdaten einfacher merken 
lassen \autocite{PwReuse}. Daraus folgt, dass es Angreifern gelingt, mit 
Brute-Force-Methoden schwache Passwörter zu erraten oder mit gestohlenen 
Zugangsdaten eines Dienstes auch bei anderen Diensten Zugang zu erhalten.
\\\\
Da es nicht leicht ist, Nutzer zu einem anderen Verhalten zu bewegen und um 
mehr Sicherheit zu gewährleisten, bieten viele Dienste die 
Zwei-Faktor-Authentisierung (2FA) an. D.h. der Nutzer bestätigt seine 
Identität mit Benutzername und Passwort (erster Faktor) sowie mit einem 
zweiten Faktor, z.B. eine sechsstellige Zahl, die von einer Smartphone-App 
generiert wird. Allerdings stellt die Zwei-Faktor-Authentisierung einen 
zusätzlichen Aufwand für den Nutzer dar. 2FA kostet den Nutzer Zeit und 
lenkt ihn von der eigentlichen Aufgabe ab, die er erledigen möchte. Durch 
diese Unbequemlichkeiten ist die 2FA nicht attraktiv genug, dass Nutzer sie 
bei vielen Diensten verwenden. Meist nutzt man es nur bei Zugängen, die man 
als wichtig ansieht, oder weil man durch den Anbieter eines Dienstes dazu 
gezwungen wird.
\\\\
Außerdem haben die meisten gängigen 2FA-Verfahren eine Schwachstelle: sie 
sind nicht resistent gegen Phishing-Websites. Also Websites, die sich als 
eine originale Website ausgeben, ohne dass es der Nutzer bemerkt. In 
Echtzeit übertragen diese Websites die Nutzereingaben an die originale 
Website und imitieren perfekt die originale Website. Je nach 2FA-Verfahren 
gibt es noch mehr Schwachstellen oder einschränkende Bedingungen, wie eine 
vorausgesetzte Verbindung zum Internet.
\\\\
Seit ca. 2019 wurde eine neue Technologie namens Passkeys veröffentlicht. Sie 
basieren auf asymmetrischer Kryptographie und ermöglichen eine Anmeldung bei 
Online-Diensten ganz ohne Passwort und zweiten Faktor. Allerdings sind sie noch 
wenig verbreitet und es werden immer noch 
Phishing-anfällige sowie geringfügig nutzerfreundliche 2FA-Verfahren verwendet. 
Daher ist es sinnvoll, im Bereich der 2FA weiterhin zu forschen.