Ziel dieser Arbeit ist die Erweiterung des 2FA-Verfahrens der Time-based One-time 
Passwords (TOTP). Dadurch sollen TOTPs in der Nutzung  nutzerfreundlicher, 
zeiteffizienter und Phishing-resistent werden. Zunächst wird analysiert, welche 
Stärken das Verfahren aufweist und wieso es für eine Erweiterung geeignet ist. Dann 
werden die Schwächen bzgl. der Nutzerfreundlichkeit und des Phishing beleuchtet und 
Lösungen präsentiert. Dies vereint sich in einem Konzept, das anschließend mit 
einem Prototyp realisiert werden soll, um zu beweisen, dass das Konzept umsetzbar 
ist. Mithilfe einer Nutzerstudie wird das System hinsichtlich seiner 
Nutzerfreundlichkeit, Nutzererfahrung und Task Completion Time untersucht. Somit 
soll die Arbeit eine Antwort darauf geben, ob der neue Prototyp der TOTPs das 
Potential zu einer bedeutsamen Verbesserung bietet. Ziel der Arbeit ist es nicht, 
diesen Prototyp als marktfähiges Produkt zu entwickeln.