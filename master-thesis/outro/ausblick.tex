Im Vergleich zu anderen wissenschaftlichen Arbeiten und kommerziellen Lösungen 
positioniert sich die Arbeit mit einer originellen Lösung in Form eines 
erweiterten Verfahrens der Zwei-Faktor-Authentisierung. Durch weitere Forschung 
und Entwicklung kann das Konzept und sein Prototyp zu einem konkurrenzfähigen 
System verfeinert werden. So wäre es auch denkbar, einen Prototypen zu bauen, 
der auf einer eigenen Instanz eines quelloffenen Browsers basiert, wodurch das 
Konzept ohne Umwege umgesetzt werden könnte und Browser-Entwickler ein Proof of 
Concept hätten. Dabei sollten auch Fragen wie die Sicherheit auf Anwendungsebene 
zwischen der Smartphone-App und dem Browser beantwortet werden. Auch von 
Interesse wäre die Betrachtung der Reaktion und Wahrnehmung des greifenden 
Phishing-Schutzes beim Nutzer. Wie kommuniziert man dem Nutzer, dass er eben 
Opfer eines Phishing-Angriffs wurde und was er nun tun sollte? Auch von 
Interesse ist, welche Entwicklung Passkeys nehmen werden. Werden sie und ggf. 
weitere ähnliche Produkte eines Tages Benutzernamen, Passwörter und somit auch 
weitestgehend die Zwei-Faktor-Authentisierung ablösen? Wird es Alternativen 
geben, die nicht unter dem Einfluss von Unternehmen wie Google, Apple oder 
Microsoft stehen werden?