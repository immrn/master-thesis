Innerhalb dieser Arbeit wurde ein Konzept entwickelt, das zum Ziel hat, 
Time-based One-time Passwords nutzerfreundlicher und schneller bei der 
Authentisierung zu machen. Das Konzept setzt voraus, in einen Browser integriert 
zu werden. Dazu wurde die Forschungsfrage gestellt, ob ein solches Konzept 
gegenwärtig ohne eine Integration seitens der Browser-Entwickler machbar ist. 
Mit einem funktionierenden Prototyp wurde gezeigt, dass es mithilfe einer 
Browser-Extension und eines Hintergrundprogramms möglich ist, dieses Konzept 
zugänglich zu machen.
Daraus bilden sich die weiteren Forschungsfragen, wie sich der Prototyp bzgl. 
der Nutzerfreundlichkeit und der Authentisierungszeit im Vergleich zum 
traditionellen Verfahren der TOTPs auswirkt. Dafür wurde eine mehrtägige Studie 
mit einem eigenen Webdienst konzipiert und entwickelt. Für die 
Nutzerfreundlichkeit wurde gemessen an den Skalen des User Experience 
Questionnaire eine deutliche Verbesserung von mindestens $0{,}3$ bis $1{,}3$ 
Punkte festgestellt (je nach Skala). Auch bei der Bewertung der System Usability 
Scale konnte eine Verbesserung von $73$ auf $78$ Punkte verzeichnet werden. Das 
qualitative Feedback der Studienteilnehmer identifiziert einige Probleme, für deren Lösung 
wiederum erste Ansätze präsentiert wurden. Die Einstellung gegenüber dem 
Prototyp war überwiegend positiv und ein Großteil der Teilnehmer würde den 
Prototyp dem traditionellen Verfahren vorziehen. Auch die Forschungsfrage zur 
Auswirkung des Prototyps auf die Authentisierungszeit im Vergleich zum 
traditionellen Verfahren kann beantwortet werden. Der Prototyp erreichte im 
Vergleich zu einer ähnlichen Studie eine Verbesserung von $15{,}1~s$ zu $8{,}
5~s$ im Median. Die genauere Betrachtung der Daten zeigt, dass für geübte Nutzer 
sogar eine Authentisierungszeit von ca. $5~s$ bis $6~s$ realistisch ist. Somit 
steht fest, dass der Prototyp für die Authentisierung nutzerfreundlicher ist und 
weniger Zeit beansprucht. Bei der Einrichtung ergeben sich Defizite, die schon 
bei der Entwicklung erwartet wurden. Allerdings konnten viele Erkenntnisse 
gewonnen werden, wie die Einrichtung mit ersten Ansätzen einfacher umgesetzt werden kann.