Jeder Proband benötigt für die Studie einen Bluetooth-fähigen Computer und ein 
Bluetooth-fähiges Android-Smartphone. Idealerweise sollten die Geräte Eigentum der 
Probanden sein, da man für gewöhnlich am besten mit seinen eigenen Geräten und 
deren Konfigurationen zurechtkommt. Zur Einführung und zur selbstständigen 
Nutzungsphase benötigen die Probanden einen Internetzugang, um den Webdienst 
aufzurufen. Um die nötigen Softwarekomponenten zu installieren, sollten sie 
Administratorrechte besitzen. Die Softwarekomponenten setzen sich zusammen aus 
der Smartphone-App (Speichern und Übertragen der TOTPs) sowie der Browser 
Extension (für Chrome) und einer Hintergrundanwendung (zum Empfangen und 
automatischen Eintragen des TOTPs). Details sind in Kap. \ref{sec: implementierung} zu finden.
\\\\
Ein weiterer Bestandteil des Experiments ist der Webdienst, der zuverlässig und 
jederzeit erreichbar sein sollte. Hier werden unter anderem die Anmeldedaten und 
Logging-Daten der Probanden gespeichert. Mithilfe der Zeitstempel in den 
Logging-Daten wird die Dauer jeder Authentisierung gemessen. Der Webdienst 
unterstützt die Anmeldung mit Benutzername und Passwort und verlangt nach dem 
ersten Login die Einrichtung der 2FA mit zeitbasierten Einmalpasswörtern. Dieser 
Einrichtungsprozess ist so aufgebaut wie jeder Einrichtungsprozess von TOTPs: 
QR-Code anzeigen, kurze Anweisungen, was der Nutzer tun muss, und ein Eingabefeld 
für das initale TOTP. Der Webdienst selbst bietet dem Probanden die Möglichkeit, 
eine fiktive Aufgabe zu erledigen, die bestenfalls einen sicherheitskritischen 
Kontext wie Finanzen hat, damit der Proband die 2FA zumindest in der Fiktion als 
sinnvoll erachtet.
\\\\
Darüber hinaus werden Materialien wie die standardisierten Fragebögen User 
Experience Questionnaire (UEQ) \autocite{Schrepp} und System Usability Scale (SUS)
\autocite{Brooke} benötigt. Beide Fragebögen werden insgesamt dreimal benötigt: 
nach der Einrichtung des neuen Verfahrens und zu Beginn der 
Abschlussveranstaltung für das neue sowie das gewöhnliche Verfahren.
\\\\
Zudem wurden den Probanden Fragen in einem Interview gestellt: einmal nach der 
Einrichtung des neuen Verfahrens und in der Abschlussveranstaltung. Der 
Fragenkatalog ist im Anhang \ref{anh: interview einrichtung} und \ref{anh: interview authentisierung} einzusehen. 

\subsection{Entscheidung für SUS und UEQ}
Es existieren verschiedene standardisierte Fragebögen, um zu ermitteln, wie 
Nutzer ihre Erfahrung mit einem System bewerten und inwiefern sie es als 
nutzerfreundlich erachten. Folgend wird begründet, wieso sich für UEQ und SUS entschieden wurde. 
\\\\
Ein sehr bekannter Fragebogen zur Nutzerfreundlichkeit / Benutzbarkeit ist die 
System Usability Scale (SUS) \autocite{Brooke}. Sie besteht aus 10 
standardisierten Aussagen über das verwendete System. Wobei ein Teilnehmer mit 
Hilfe einer 5-stufigen Skala zu jeder Aussage seine Zustimmung oder seinen 
Widerspruch äußert. Aus den Antworten wird eine Punktzahl berechnet, die dann 
eine Aussage über die Nutzerfreundlichkeit eines Systems treffen soll. Das 
Ergebnis liegt immer in einem Bereich von 0 bis einschließlich 100 Punkte, wobei 
100 Punkte für die bestmögliche und 0 Punkte für die schlechtestmögliche 
Nutzerfreundlichkeit stehen. Als durchschnittliche Bewertung werden 68 Punkte 
angesehen und die prozentuale Wertigkeit der Punkte verläuft nicht linear, sondern als 
S-Kurve \autocites{Sauro}[zitiert nach][]{BrookeRetro}. Die SUS eignet sich, um zwischen verschiedenen 
Systemen zu vergleichen, da ihre zu bewertenden Aussagen allgemein verfasst sind.
Prinzipiell kann der Vergleich zu den Studienergebnissen von Reese et al. \autocite{Reese} angestellt werden. Am Ende des SUS-Fragebogens wird eine 11. Frage gestellt, die separat ausgewertet wird. Hier geben die Probanden auf einer $7$-stufigen Skala an, wie nutzerfreundlich sie das System insgesamt bewerten \autocite{SUS11}. Die Stufen werden mit Aussagen wie \glqq Das Schlechteste, was man sich vorstellen kann\grqq{} über \glqq Ok\grqq{} bis hin zu \glqq Das Beste, was man sich vorstellen kann\grqq{} versehen. Diese Aussagen werden entsprechend der Erkenntnisse von \textcite{SUS11} einer SUS-Punktzahl zugeordnet. Mithilfe dieser Ergebnisse wird verglichen, ob die tatsächlichen SUS-Berwertungen der empfundenen Nutzerfreundlichkeit annährend entsprechen.
\\\\
Des Weiteren wird der User Experience Questionnaire (UEQ) Fragebogen genutzt. Er 
dient dazu, ein System bzgl. der Nutzerfreundlichkeit und 
Nutzererfahrung zu bewerten. Der UEQ besteht aus 26 Begriffspaaren, wobei sich 
jedes Paar aus jeweils zwei gegensätzlichen Adjektiven zusammensetzt. Mit einer 
7-stufigen Skala muss der Proband für jedes Begriffspaar seine Zustimmung zu 
einem der Begriffe in Bezug auf das System äußern. Setzt er ein Kreuz in der 
Mitte der Skala, stimmt er für beide Begriffe gleichermaßen. Diese 26 Paare 
werden unterteilt in sechs Skalen \autocite[2]{Schrepp}:
\begin{itemize}
    \item Attraktivität: Gesamteindruck über das System, mögen Nutzer das System oder nicht?
    \item Durchschaubarkeit: Gewöhnt man sich leicht an das System, ist es leicht zu erlernen?
    \item Effizienz: Kann die Aufgabe ohne unnötigen Aufwand gelöst werden?
    \item Steuerbarkeit: Beherrscht der Nutzer das System?
    \item Stimulation: Ist die Nutzung des Systems spannend und motivierend?
    \item Originalität: Ist das System innovativ und kreativ? Weckt es das Interesse?
\end{itemize}
Somit wird analysiert, ob in einer Skala Stärken oder Defizite vorliegen. Dies kann 
man sowohl für ein einzelnes System (hier die Einrichtung von Blue TOTP) als auch zum Vergleich 
zweier Systeme nutzen (hier das traditionelle TOTP-Verfahren gegenüber Blue TOTP).