An der TU Bergakademie Freiberg wurde öffentlich die Studie über die tägliche 
Rundmail beworben. Voraussetzung für die Teilnehmer war, dass sie bereits eine 
TOTP-App genutzt haben. D.h. sie sollten Kenntnis darüber haben, wie man das 
gewöhnliche TOTP-Verfahren einrichtet und bei der 2FA benutzt. Dies ist 
notwendig, damit sie das neue mit dem gewöhnlichen Verfahren vergleichen 
können. Des Weiteren müssen sie über ein Smartphone mit Android 10.0 oder höher 
verfügen sowie über einen Bluetooth-fähigen Computer mit Windows 10 oder 11. 
Eine weitere Beschränkung war, dass die Studie nur auf deutsch durchgeführt 
wurde, da die Benutzeroberfläche der Softwarekomponenten sowie die gesamten 
Unterlagen und Fragebögen zur Studie nur in deutsch verfasst sind.
Demographischen Einschränkungen bei der Auswahl der Teilnehmer bedarf es nicht.
Für die Teilnahme an der Studie hatten die Probanden einen Zeitaufwand von ca. 
2-3h und erhielten für die gesamte Studie eine Vergütung von 40 Euro.
\\\\
Insgesamt haben 11 Personen teilgenommen, davon 6 weibliche und 5 männliche. 
Rund 73\% der Teilnehmer gehören der Altersgruppe von 20-30 Jahren an. Die 
meisten Teilnehmer sind Studenten. Alle Teilnehmer besitzen mindestens das Abitur o.ä. als höchsten schulichen Abschluss.
\\\\
Bezüglich der Vorerfahrung mit TOTPs wurden die Teilnehmer ebenfalls befragt. 
Durchschnittlich haben sie 2 Dienste genutzt (jeder mind. einen), bei denen sie 
TOTPs als zweiten Faktor nutzen. Vier Probanden gaben an, TOTPs nur 1-3 mal pro Monat zu 
nutzen, während sechs angaben es 2 mal pro Woche oder häufiger zu nutzen. Bei der 
Frage, ob sie sich an die Einrichtung des TOTP-Verfahrens erinnern, konnten nur 
wenige beschreiben, dass man einen QR-Code scannt und anschließend das erste 
TOTP eingibt. Nachdem es ihnen erklärt wurde, erinnerten sich alle Teilnehmer wieder an 
die Einrichtung und sagten aus, dass sie die Einrichtung nicht 
als schwierig empfanden. Zwei Teilnehmer gaben an, noch nie aktiv eine Browser-Extension genutzt zu haben.