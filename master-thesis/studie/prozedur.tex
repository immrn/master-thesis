Bevor interessierte Personen zur Studie zugelassen wurden, mussten sie eine 
Online-Umfrage mit vier Fragen durchlaufen. Dabei wurde erfragt, ob sie 
Vorerfahrung mit zeitbasierten Einmalpasswörtern und einer entsprechenden 
Authenticator App haben. Außerdem wurde überprüft, ob ihr Smartphone eine 
gewisse Bluetooth-Funktion unterstützt (siehe Kap. \ref{sec: implementierung architektur}). Am Ende 
wurden sie noch gebeten, den Chrome-Browser zu installieren. Waren alle 
Voraussetzungen erfüllt, wurden sie zur Einführung eingeladen. Die Einführung 
und der Abschluss fanden in Einzelsitzungen statt. So konnten die Probanden 
bei der Einrichtung besser beobachtet werden und der Versuchsleiter konnte jedem 
Probanden mit ungeteilter Aufmerksamkeit behilflich sein. Da es Einzelsitzungen waren, war es nicht möglich, dass die Antworten eines Probanden durch 
andere Probanden beeinflusst wird.