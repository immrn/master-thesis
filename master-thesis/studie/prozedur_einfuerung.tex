Zu Beginn der Einführungsveranstaltung haben die Teilnehmer die 
Teilnehmerinformation gelesen, falls noch nicht geschehen, und füllten 
anschließend die Einverständniserklärung aus. Danach wurden sie noch einmal 
kurz über den Ablauf der gesamten Studie aufgeklärt. Dabei wurden sie bestärkt, 
dass sie nichts falsch machen können und bei Problemen jederzeit den 
Versuchsleiter ansprechen sollen. Dann wurde mit der Aufnahme der 
demographischen Daten sowie der Vorkenntnisse zu TOTPs begonnen. Dies geschah 
teilweise automatisiert mit LimeSurvey\footnote{\href{https://www.limesurvey.org/}{https://www.limesurvey.org/}}, teilweise als Interview. Die Teilnehmer wurden 
gefragt, ob ihnen das Prinzip der 2FA bekannt ist.
\\\\
Im nächsten Abschnitt installierten die Teilnehmer gemeinsam mit dem 
Versuchsleiter die notwendigen Programme: ggf. Chrome, die Browser-Extension 
und deren Hintergrundanwendung sowie die Smartphone-App. Da es sich bei der 
Implementierung um einen Prototyp handelt, sind die App, die Extension und 
deren Hintergrundanwendung nicht in den offiziellen Software Stores (Google 
Play Store, Chrome Web Store, Microsoft Store) veröffentlicht und auch nicht 
signiert. D.h. die Teilnehmer mussten für ihr Android-Gerät die App über eine  
\glqq Android Package Kit\grqq-Datei (APK) installieren und fremde Quellen 
zulassen. Für Windows war es ähnlich. Die Extension musste über den 
Entwicklermodus von Chrome installiert werden. Dadurch war der gesamte 
Installationsprozess verhältnismäßig sehr aufwändig und daher kein Bestandteil 
der Beurteilung der Probanden. Nach der Installation haben die Probanden die 
Extension und App kurz geöffnet, um die Onboardings zu überspringen, da diese 
gerade durchgeführt wurden. Daraufhin haben sich die Probanden beim Webdienst 
registriert. Ab diesem Punkt wurde erwähnt, dass die Probanden alles vergessen 
sollten, was sie bis jetzt durchlaufen haben. Somit sollte sichergestellt werden, dass anschließend nur die Einrichtung der 2FA mit Blue TOTP bewertet wird und nicht die Installation der Programme.
\\\\
Den Probanden wurde erklärt, dass sie nun mit der Einrichtung des neuen 
Authentisierungsverfahrens beginnen werden, wobei sie die eben installierten 
Programme (App und Extension) nutzen sollen. Ebenso wurde ihnen gesagt, dass sie diesen Prozess 
anschließend bewerten werden. Die Probanden haben sich beim Webdienst 
angemeldet und eine Anleitung zur Einrichtung der Zwei-Faktor-Authentisierung 
gesehen. 
Zu diesem Zeitpunkt startet der Versuchsleiter unauffällig die Stoppuhr. Dann 
mussten die Probanden den zusätzlichen Anweisungen in der App und der Browser 
Extension folgen. Nur bei großen Schwierigkeiten schritt der Versuchsleiter 
ein, um sie auf den richtigen Pfad zu führen. Während der Einrichtung wurden 
alle Probleme oder Auffälligkeiten notiert. Mit erfolgreichem Abschluss der 
Einrichtung wurde die Stoppuhr angehalten und die Zeit notiert.
\\\\
Anschließend haben die Teilnehmer UEQ und SUS bzgl. der Einrichtung  
ausgefüllt. Danach wurde das Interview mit ihnen geführt (siehe Anhang \ref {anh: interview einrichtung}).
Das Interview dient dazu, die so eben erlebten Erfahrungen, Gedanken und Ideen 
des Probanden zu äußern. Bis zum Ende des Interviews (also auch nicht bei SUS und 
UEQ) wusste der Proband noch nicht, worin sich das eben untersuchte Verfahren vom 
gewöhnlichen Verfahren unterscheidet. Denn der Proband hat sich bis zu diesem 
Zeitpunkt noch nicht mithilfe des neuen Verfahrens authentisiert. Somit sind 
seine Aussagen nicht davon beeinflusst, wie gut oder schlecht das System in 
seiner eigentlichen Nutzung (der 2FA) funktioniert.
\\\\
Am Ende hat der Proband noch einen Testlauf für die kommenden Tage 
durchgeführt und so die eigentliche Nutzung des neuen Verfahrens kennengelernt. D.h. er hat sich zunächst beim Webdienst erneut angemeldet. 
Daraufhin hat er mit Blue TOTP erstmals das Einmalpasswort automatisch in die 
Website übertragen lassen. Dann hat er auf der Website noch eine fiktive 
Überweisung getätigt.
Zum Schluss wurden noch Formalitäten wie der Termin zum Abschlussmeeting geklärt 
und der erste Teil der Vergütung in Höhe von 15 Euro ausgezahlt. Die Einführung 
dauerte i.d.R. eine Stunde.