Nun hat der Proband die Aufgabe, sich täglich für sechs Tage in Folge beim 
Webdienst anzumelden und eine fiktive Überweisung zu tätigen. Dies geschah von 
zu Hause aus oder unterwegs. Dazu muss er nach der Anmeldung mit Benutzername 
und Passwort immer den zweiten Faktor mit dem neuen Verfahren automatisch in 
die Website übertragen lassen. Um den Probanden nicht unter Druck zu setzen, 
wurde nicht kommuniziert, dass die benötigte Zeit zur Authentisierung gemessen 
wurde. Immer um 1:00 Uhr morgens wurde den Teilnehmern ihre Aufgabe per E-Mail 
zugesendet. Wer bis 19:00 seine Aufgabe noch nicht erledigt hatte, bekam eine 
Erinnerungsmail. Wenn ein Teilnehmer seine Aufgabe nicht erfüllt hat, sollte er 
am Folgetag einmal früh und dann abends die Aufgabe erledigen.