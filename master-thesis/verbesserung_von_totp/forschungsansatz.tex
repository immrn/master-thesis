Das vorgestellte Konzept zur Verbesserung von zeitbasierten Einmalpasswörtern 
soll als aktuelle Implementierung umgesetzt werden. Während es wünschesnwert ist, dass Browser-Anbieter das Konzept als neuen Standard unterstützen, erfolgt die Überprüfung der Machbarkeit in Form eines Prototypen. 
D.h. es soll eine Lösung geschaffen werden, die nicht erst in entfernter Zukunft 
nutzbar ist, sondern jetzt.
Anhand dieser Implementierung wird untersucht, ob die beschriebenen Änderungen am 
gewöhnlichen \mbox{TOTP-Verfahren} einen Mehrwert für den Nutzer generieren. Es stellen sich somit die folgenden Forschungsfragen:
\begin{itemize}
    \item Wie kann eine eine solche gegenwärtige Implementierung mit einem Schutz gegen \\MITM-Phishing realisiert werden?
    \item Welche Auswirkungen hat die Implementierung auf die Nutzererfahrung und die Nutzerfreundlichkeit?
    \item Welche Auswirkungen hat die Implementierung auf die Task Completion Time?
\end{itemize}
Das Konzept und die zugehörige Implementierung zielen darauf ab, Schutz vor Phishing zu bieten. Allerdings wird dieser Aspekt weder technisch noch bzgl. der Nutzererfahrung weiter untersucht.
\\\\
Daraus werden folgende Hypothesen abgeleitet:
\begin{itemize}
    \item[(a)] Der Prototyp des neuen Verfahrens für zeitbasierte Einmalpasswörter ist nutzerfreundlicher als das gewöhnliche TOTP-Verfahren.
    \item[(b)]  Der Prototyp des neuen Verfahrens für zeitbasierte Einmalpasswörter benötigt bzgl. der Authentisierungsvorgänge weniger Zeit als das gewöhnliche TOTP-Verfahren.
\end{itemize}
Um den Forschungsfragen nachzukommen, wurde eine Studie mit Probanden geplant. 
Dazu muss das neue TOTP-Verfahren unterschieden werden in seine Einrichtung und 
seine eigentliche Nutzung beim Login. Das heißt, es ist sinnvoll die Studie wie \textcite{Reese} in einen Termin zur Einrichtung, einer mehrtägigen Nutzungsphase und einen Abschlusstermin aufzuteilen. Voraussetzung zur Durchführung der Studie ist die funktionierende Implementierung 
des neuen TOTP-Verfahrens, die gegenwärtig funktioniert und nicht voraussetzt, dass Browser das neue Konzept als Standard unterstützen.
Die Auswahl der Probanden soll sich auf Nutzer beschränken, die bereits Erfahrung 
mit zeitbasierten Einmalpasswörter haben, speziell mit den TOTP-Apps (nicht SMS 
oder Email). Durch Interviews und Fragebögen kann so ermittelt werden, inwiefern 
die Implementierung des neuen TOTP-Verfahrens eine Verbesserung zu dem 
gewöhnlichen TOTP-Verfahren darstellt.

\paragraph*{Erwartungen an die Studie}
\mbox{} \vspace{0.1cm} \\
Da nun der Forschungsansatz vorgestellt wurde, stellt sich die Frage, welche 
Erwartungen an die Studie gestellt werden. Was soll die Studie bewirken
\\\\
Zum einen gibt es die Zeitmessung der Einrichtungs- und 
Authentisierungsvorgänge. Allerdings wird sie ohne eine zweite Probandengruppe, 
die das gewöhnliche TOTP-Verfahren testet, keinen echten Vergleichswert haben. 
Dennoch kann ihre Dimension als Vergleich für ähnliche Studien genutzt werden. 
Die Zeitmessung basiert auf einem Logging: also was tut der Nutzer beim 
Webdienst. So können auch fehlgeschlagene Anmeldungen beziffert werden. D.h. man 
kann keine Aussage darüber treffen, ob das neue TOTP-Verfahren nun tatsächlich 
schneller als das gewöhnliche ist. Mit einer zweiten Probandengruppe für das 
gewöhnliche Verfahren könnte man eine Aussage treffen. Mit den 
Fragen im Interview kann zumindest festgestellt werden, ob die Teilnehmer das 
neue TOTP-Verfahren schneller oder langsamer empfinden und aus welchen Gründen.
\\\\
Bezüglich der Nutzerfreundlichkeit / Gebrauchstauglichkeit und der 
Nutzererfahrung geben die SUS und der UEQ eine gute Vorstellung davon, wie die 
beiden Verfahren sich hinsichtlich des Authentisierungsvorgangs unterscheiden. 
Auch gibt das Interview in der Abschlussveranstaltung Aufschluss darüber, aus 
welchen Gründen die Teilnehmer eines der Systeme bevorzugen. Die Fragebögen (SUS 
und UEQ) zur Einrichtung sind eher informativ, da bereits vermutet wird, dass das 
Einrichtungsverfahren mit der bereitgestellten Implementierung etwas aufwendiger 
ist als die Einrichtung des gewöhnlichen Verfahrens. Die Fragen 
und Beobachtungen bezüglich der Einrichtung dienen dazu, die Probleme der zu 
identifizieren und neue Ideen zu sammeln, um den Vorgang zukünftig leichter 
zu gestalten.