Im folgenden Abschnitt werden Probleme des TOTP-Verfahrens und Ideen zu deren 
Lösung vorgestellt. Diese Lösungen vereinen sich in einem Konzept zur 
Verbesserung von TOTPs.

\paragraph*{TOTP-App als Grundlage}
\mbox{} \vspace{0.1cm} \\
Zunächst stellt sich die Frage, wieso das TOTP-Verfahren und insbesondere die 
TOTP-Apps für Erweiterungen geeignet sind. Zuerst sei genannt, dass das 
Smartphone als der zweite Faktor fungiert. D.h. man benötigt keinen neuen 
Gegenstand wie einen Security-Key für die 2FA. Diese TOTP-Apps sind unabhängig 
von den Diensten, bei denen sich ein Nutzer authentisieren möchte. Ein Dienst muss nur das TOTP-Verfahren unterstützen, ansonsten hat er 
keinen Einfluss auf die App. Somit kann der Nutzer seine 2FA-Vorgänge mithilfe 
einer App erledigen, anstatt für jeden Dienst eine eigene App zu nutzen. Das ist 
nämlich der Fall beim Push-Verfahren. Wobei es auch Push-Verfahren wie Authy One 
Touch gibt, bei denen eine App für mehrere Dienste genutzt werden kann. 
Andererseits benötigt man mit den TOTP-Apps keinen Mobilfunkempfang oder 
Internetzugang, da das Einmalpasswort von der App selbst berechnet wird.  
Betrachtet man nun die Probleme des 
TOTP-Verfahrens, lassen sich diese mit einigen Erweiterungen lösen.

\paragraph*{Ablesen und Eintippen des TOTPs}
\mbox{} \vspace{0.1cm} \\
Ein wesentliches Problem ist das Ablesen und Eintippen des 
Einmalpassworts. Die App generiert das TOTP und der Nutzer muss es bei jedem 
Authentisierungsvorgang in die Website eingeben. Zum einen ist das 
zeitintensiv und unbequem \autocite{Reese}, zum anderen lenkt es auch von der Aufgabe ab, die der 
Nutzer eigentlich erledigen möchte. Dazu kommt, dass potentiell die Ziffern des 
Einmalpasswort falsch abgelesen bzw. falsche Ziffern eingegeben werden.

\paragraph*{Automatische Übertragung des TOTPs}
\mbox{} \vspace{0.1cm} \\
Es ist fraglich, wieso der Nutzer überhaupt das TOTP übertragen soll. 
Stattdessen kann die App das TOTP an den Browser senden und dieser schreibt es 
automatisch in das Eingabefeld. Also benötigt man eine geeignete Technologie zur 
Datenübertragung, die von den meisten Smartphones und Computern unterstützt wird 
oder leicht nachzurüsten ist. Infrage kommen dabei NFC, Bluetooth und prinzipiell 
auch der Weg über einen Server im Internet. QR-Codes sind unpassend, da sie nur 
unidirektional kommunizieren und der Nutzer aktiv werden muss. NFC ist zwar in 
den meisten modernen Smartphones integriert, aber wird nicht von jedem Laptop und 
noch seltener von Desktop-PCs unterstützt. Man kann NFC-Chips in 
Form von USB-Geräten an den PC anschließen. Allerdings sind sie für den mobilen 
Gebrauch eher unhandlich. Der Weg, das TOTP über einen Server zu übertragen, 
würde wieder einen Internetzugang erfordern und, wie später noch ausgeführt wird, 
die Angriffsfläche für MFA-Fatigue öffnen. Bluetooth dagegen wird von den meisten 
modernen Smartphones und Laptops unterstützt. Gegebenenfalls kann es auch mit kleinen 
erschwinglichen USB-Geräten nachgerüstet werden. Zudem sendet Bluetooth je nach 
Konfiguration und Bedingungen über Reichweiten von wenigen Metern bis hin zu 
mehreren Hundert Metern \autocite{btEstimator}, während NFC nur mit Reichweiten von bis zu $2~cm$ \autocite{nfcForum} 
agiert. Bluetooth ist auch bereits in den Browsern Chrome, Edge und Opera 
integriert und nennt sich Web Bluetooth \autocite{webBt}. Daher fällt die 
Entscheidung auf Bluetooth als Funktechnologie.

\paragraph*{Nutzer als letzte Sicherheitsinstanz}
\mbox{} \vspace{0.1cm} \\
Einerseits sollte das Einmalpasswort nicht ohne eine Interaktion des Nutzers an 
den Browser gesendet werden. Denn auch wenn die Verbindung zwischen der App und 
dem Browser sicher ist, könnte jemand mit Zugang zum Computer des Nutzers 
versuchen, sich bei einem Dienst anzumelden, während der Nutzer dies nicht 
bemerkt. Andererseits ist es sogar die Überlegung wert, doch ohne jegliche 
Nutzerinteraktion das TOTP an Browser zu senden. Voraussetzung wäre natürlich, 
dass die Verbindung zwischen App und Browser Ende-zu-Ende verschlüsselt, 
authentisiert und resistent gegen jegliche kompromittierende Angriffe ist, wie 
bspw. Replay-Attacken oder Man-In-The-Middle-Attacken beim Schlüsselaustausch. 
Ist das gegeben, könnte theoretisch nur der Computer bzw. Browser das TOTP vom 
Smartphone verlangen. Andererseits könnte es ein Risiko sein, wenn der Computer 
bzw. der Browser mit Malware infiziert ist. D.h. es bedarf einer Abwägung aus Sicherheit und Komfort: soll die App einfach TOTPs an einen vertrauten Browser senden dürfen oder soll der Nutzer jedes TOTP freigeben, bevor die App es an den Browser sendet.

\paragraph*{Transformation zum Push-Verfahren}
\mbox{} \vspace{0.1cm} \\
Ein weiteres Problem bei den gewöhnlichen TOTP-Apps ist, dass der Nutzer die App 
erst suchen, öffnen und den entsprechenden Account aus einer Liste wählen muss, 
um an das Einmalpasswort zu gelangen. Das kann behoben werden, indem man die 
TOTP-App mithilfe der Bluetooth-Funktionalität in eine App verwandelt, die aus 
Sicht des Nutzers einer App mit Push-Verfahren gleicht. Sobald der Browser 
erkennt, dass der Nutzer das TOTP eingeben muss, sendet er der App einen Befehl, 
damit diese eine Benachrichtigung anzeigen kann. So kann der Nutzer auf die 
Benachrichtigung tippen und muss nur noch zustimmen, dass das TOTP an den Browser 
übertragen werden soll. Dadurch könnte das neue Konzept zusätzlich Zeit bei der 
Authentisierung einsparen  (vergl. Abb. \ref{fig: reese zeiten} Push vs. 
TOTP) und weniger Interaktionen vom Nutzer verlangen. Das Problem, das Push-Apps der MFA-Fatigue ausgesetzt sind, gilt hier nur noch für den lokalen Bereich, wenn der Angreifer in der Lage ist, das Smartphone des Nutzers per Bluetooth zu erreichen.

\paragraph*{Schutz gegen Phishing}
\mbox{} \vspace{0.1cm} \\
Nachdem nun einige Ansätze zu einer verbesserten Nutzerfreundlichkeit und einem 
geringeren Zeitaufwand vorgestellt wurden, muss auch das Problem des fehlenden 
Schutzes vor Phishing (siehe Kap. \ref{sec: phishing}) gelöst werden. Prinzipiell 
muss der Browser (nicht die Website) als vertrauenswürdige Partei angesehen 
werden. Richtet der Nutzer dieses neue Verfahren bei einem Webdienst ein, dann 
soll der Browser den Benutzernamen und die Domain der Website an die 
Smartphone-App senden. Den Benutzernamen ermittelt der Browser, wenn sich der 
Nutzer beim Webdienst anmeldet. Die App erhält bei der Einrichtung das Geheimnis, 
indem sie den QR-Code auf der Website scannt. Die App speichert den Benutzernamen 
und die Domain zusammen mit dem Geheimnis. Meldet sich der Nutzer nach der 
Einrichtung an, empfängt die App vom Browser die Domain der Website und den 
Benutzernamen. Die App prüft, ob sie diese Domain bereits kennt. Ist dies der 
Fall, zeigt die App dem Nutzer die Benachrichtigung an und sendet bei dessen 
Zustimmung das entsprechende TOTP an den Browser. Kennt die App die Domain nicht, 
dann sollte sie den Nutzer warnen, dass er sich auf einer Phishing-Website 
befindet.

\paragraph*{Kein Zeitdruck und unnötiges Warten}
\mbox{} \vspace{0.1cm} \\
Zuletzt seien noch zwei zeitkritische Probleme von Interesse. Generell entfernt 
das neue Konzept den wahrnehmbaren Zeitdruck. Der Nutzer muss das TOTP nicht mehr 
abtippen und sieht auch keinen Countdown mehr, der eventuell Stress auslösen 
kann. Außerdem entfällt das Problem, dass der Nutzer warten muss, wenn das TOTP 
nur noch wenige Sekunden gültig ist (z.B. 5s). In diesem Fall wartet der Nutzer oft 
auf das nächste Einmalpasswort, weil er es in der verbleibenden Zeit nicht in den 
Browser übertragen könnte. Man könnte vermuten, dass dieses Problem in anderer 
Weise auch für das neue Konzept gilt. Zum Beispiel könnte die App das 
Einmalpasswort genau dann senden, wenn es nur noch 1s gültig ist. Dann wird es 
zwar automatisch in die Website eingegeben, aber falls die Übertragung und die 
Bestätigung der Eingabe mehr als $1~s$ dauern, wäre das Einmalpasswort nicht mehr 
gültig. Aber der RFC 6238 empfiehlt, dass der Webdienst das TOTP des aktuellen 
30s-Zeitfensters sowie das TOTP im Zeitfenster davor und danach akzeptiert. Also 
ist das Einmalpasswort aus dem eben genannten Beispiel noch ca. $30~s$ gültig.