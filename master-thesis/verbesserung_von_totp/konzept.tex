Das Konzept zur Verbesserung adressiert die Probleme des TOTP-Verfahrens und 
beabsichtigt, die vorangegangenen Ideen zu vereinen, so dass es als neuer Standard für 
TOTPs verwendet werden kann.
Damit wird die Vision geschaffen, dass aktuelle Browser und 
TOTP-Apps dieses Konzept in Zukunft unterstützen. Vor allem ist der Browser  essentiell, da er die kontrollierende Partei verkörpert. Das Konzept soll folgende Ziele verwirklichen:
\begin{itemize}
    \item Sicherheit gegen Phishing
    \item Bessere Nutzererfahrung und mehr Nutzerfreundlichkeit durch reduzierte Arbeitslast und weniger Interaktionen
    \item schnellere Authentisierung durch die automatische Übertragung des TOTPs
\end{itemize}
Das Konzept unterteilt sich wie auch das übliche TOTP-Verfahren in Einrichtung und Authentisierungsvorgang.
